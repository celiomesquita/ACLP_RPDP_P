% !Rnw weave = knitr
%%% DO NOT EDIT the .tex file directly since it is generated from the .Rnw
%%% sources.
%\VignetteEngine{knitr::knitr}
%\VignetteIndexEntry{irace package: User Guide}
%\VignetteDepends{knitr}
%\VignetteCompiler{knitr}
\synctex=1
\RequirePackage[dvipsnames]{xcolor}
\documentclass[a4paper,english]{article}
\usepackage[a4paper]{geometry} % It saves some pages
\usepackage[utf8]{inputenc}
\usepackage[T1]{fontenc}
\usepackage[english]{babel}
\usepackage{ifthen}
\newboolean{Release}
\setboolean{Release}{true}
\usepackage{calc}
\usepackage{afterpage}
\usepackage{algorithm,algorithmic}
\usepackage{booktabs}
\usepackage{tabularx}
\usepackage{xspace}
\usepackage{amsmath,amssymb}
\usepackage{relsize}
\usepackage{fancyvrb}
\usepackage[hyphens]{url}
\usepackage{hyperref}
\usepackage[numbers]{natbib}
\usepackage[nottoc]{tocbibind}
%% For autoref
\hypersetup{
	colorlinks,
	linkcolor={red!50!black},
	citecolor={blue!50!black},
	urlcolor={blue!70!black}
}
\addto\extrasenglish{%
	\def\sectionautorefname{Section}
	\let\subsectionautorefname\sectionautorefname
	\let\subsubsectionautorefname\sectionautorefname
}
\usepackage[titletoc, title]{appendix}
% Fix use with \autoref
\newcommand*{\Appendixautorefname}{Appendix}
\usepackage{tocloft}
\setlength{\cftsubsecnumwidth}{3em}% Set length of number width in ToC for \subsection
\usepackage[inline]{enumitem}
\setlist[enumerate]{leftmargin=*,widest=00}
\setlist[itemize]{leftmargin=1.5em}

%% FIXME: listing is very limited, we should use 'minted'
\usepackage{listings}
\lstdefinestyle{BashInputStyle}{
	language=bash,%
	basicstyle=\ttfamily,%
	numbers=none,%
	frame=tb,%
	rulecolor=\color{lightgray},
	%  framesep=1ex,
	framexleftmargin=1ex,
	columns=fullflexible,%
	backgroundcolor=\color{yellow!05},%
	linewidth=\linewidth,%
	% xleftmargin=1\linewidth,%
	identifierstyle=\color{darkgray},%
	keywordstyle=\color{darkgray},%
	keywordstyle={[2]\color{Cyan}},%
	keywordstyle={[3]\color{olive}},%
	stringstyle=\color{MidnightBlue},%
	commentstyle=\color{RedOrange},%
	morestring=[b]',%
	showstringspaces=false
}

\DefineVerbatimEnvironment{Code}{Verbatim}{}
\DefineVerbatimEnvironment{CodeInput}{Verbatim}{fontshape=rm}
\DefineVerbatimEnvironment{CodeOutput}{Verbatim}{}
\newenvironment{CodeChunk}{}{}


\newcommand{\IRACEHOME}[1]{\hyperlink{irace_home}{\path{$IRACE_HOME}}\path{#1}}

\providecommand{\keywords}[1]{\textbf{\textit{Index terms---}} #1}

% Simple font selection is not good enough.  For example, |\texttt{--}|
% gives `\texttt{--}', i.e., an endash in typewriter font.  Hence, we
% need to turn off ligatures, which currently only happens for commands
% |\code| and |\samp| and the ones derived from them.  Hyphenation is
% another issue; it should really be turned off inside |\samp|.  And
% most importantly, \LaTeX{} special characters are a nightmare.  E.g.,
% one needs |\~{}| to produce a tilde in a file name marked by |\file|.
% Perhaps a few years ago, most users would have agreed that this may be
% unfortunate but should not be changed to ensure consistency.  But with
% the advent of the WWW and the need for getting `|~|' and `|#|' into
% URLs, commands which only treat the escape and grouping characters
% specially have gained acceptance
\makeatletter
\DeclareRobustCommand\code{\bgroup\@makeother\_\@makeother\~\@makeother\$\@noligs\@codex}
\def\@codex#1{\texorpdfstring%
	{{\normalfont\ttfamily\hyphenchar\font=-1 #1}}%
	{#1}\egroup}
\makeatother

\let\proglang=\textsf
\newcommand{\pkg}[1]{{\fontseries{b}\selectfont #1}}
\newcommand{\aR}{\proglang{R}\xspace}
\newcommand{\MATLAB}{\proglang{MATLAB}\xspace}
\newcommand{\eg}{e.g.,\xspace}
\newcommand{\SoftwarePackage}{\pkg}
\newcommand{\ACOTSP}{\SoftwarePackage{ACOTSP}\xspace}

%% How to use this command:
% Parameter with one short switch: \defparameter[short]{paramName}{long}{default}
% Parameter without short switch:  \defparameter{paramName}{long}{default}
% Parameter without switch:  \defparameter{paramName}{}{default}
\newcommand{\defparameter}[4][]{%
	\item[\code{#2}]\hypertarget{opt:#2}{} ~~ %
	\ifthenelse{\equal{#3}{}}{}{%
		\emph{flag:} %
		\ifthenelse{\equal{#1}{}}{}{%
			\code{-#1}~~~\emph{or}~~~}%
		\code{-{}-#3} ~~ }%
	\emph{default:}~\texttt{#4} \\
}
\newcommand{\parameter}[1]{\hyperlink{opt:#1}{\code{#1}}}
%\usepackage{showlabels}
%\showlabels{hypertarget}

\newcommand{\irace}{\pkg{irace}\xspace}
\newcommand{\Irace}{\pkg{Irace}\xspace}
\newcommand{\race}{\pkg{race}\xspace}
\newcommand{\FRACE}{\text{F-Race}\xspace}
\newcommand{\IFRACE}{\text{I/F-Race}\xspace}
\newcommand{\PyImp}{\pkg{PyImp}\xspace}
\newcommand{\iraceversion}{\Sexpr{packageVersion("irace")}}

\newcommand{\Niter}{\ensuremath{N^\text{iter}}\xspace}
\newcommand{\Nparam}{\ensuremath{{N^\text{param}}}\xspace}
\newcommand{\iter}{\ensuremath{j}\xspace}
\newcommand{\Budget}{\ensuremath{B}\xspace}
\newcommand{\Budgetj}{\ensuremath{\Budget_{\iter}}\xspace}
\newcommand{\Bused}{\ensuremath{\Budget_\text{used}}\xspace}
\newcommand{\Ncand}[1][]{\ensuremath{N_{#1}}\xspace}
\newcommand{\Mui}{\ensuremath{\mu_{\iter}}\xspace}
\newcommand{\Nmin}{\ensuremath{N^\text{min}}\xspace}
\newcommand{\Nsurv}{\ensuremath{N^\text{surv}}\xspace}
\newcommand{\Nelite}{\ensuremath{N^\text{elite}}\xspace}
\newcommand{\Nnew}{\ensuremath{N^\text{new}}\xspace}

\newcommand{\bmax}{\ensuremath{b^\text{max}}\xspace}
\newcommand{\bmin}{\ensuremath{b^\text{min}}\xspace}
\newcommand{\Celite}{\ensuremath{\Theta^\text{elite}}\xspace}

\ifthenelse {\boolean{Release}}{%
	\newcommand{\MANUEL}[1]{}
	\newcommand{\LESLIE}[1]{}
	\newcommand{\THOMAS}[1]{}
}{%
	\newcommand{\MANUEL}[1]{{\footnotesize\noindent\textbf{\color{red}[~MANUEL: #1~]}}}
	\newcommand{\LESLIE}[1]{\footnote{\noindent\textbf{[ LESLIE: #1 ]}}}
	\newcommand{\THOMAS}[1]{\footnote{\noindent\textbf{[ THOMAS: #1 ]}}}
}
\newcommand{\hide}[1]{}

\usepackage{tcolorbox}
\newcommand{\infoicon}{%
	\parbox[c]{0.75cm}{\includegraphics[keepaspectratio=true,width=0.75cm]{light-bulb-icon}}%
	\hspace{1em}}
\newcommand{\warningicon}{%
	\parbox[c]{0.75cm}{\includegraphics[keepaspectratio=true,width=0.75cm]{Warning-icon}}%
	\hspace{1em}}

\definecolor{LightGray}{RGB}{193,193,193}
\definecolor{LightYellow}{RGB}{253,247,172}

\newlength\macroiconwidth
\newenvironment{xwarningbox}{%
	\setlength{\fboxrule}{3.0\fboxrule}%
	\setlength{\fboxsep}{0\fboxsep}%
	\begin{tcolorbox}[colback=LightYellow,colframe=LightGray,boxrule=\fboxrule,boxsep=\fboxsep]%
		\infoicon%
		\settowidth{\macroiconwidth}{\infoicon}%
		\begin{minipage}[c]{\columnwidth - \macroiconwidth - 2.0\fboxrule - 2.0\fboxsep}
			\raggedright\footnotesize
			%
		}{%
		\end{minipage}
	\end{tcolorbox}
	%
}

\begin{document}
	
	<<include=FALSE>>=
	library(knitr)
	@
	
	\author{Manuel L\'opez-Ib\'a\~nez, Leslie P\'erez C\'aceres, J\'er\'emie Dubois-Lacoste,\\
		Thomas St\"utzle and Mauro Birattari
		\\IRIDIA, CoDE, Universit\'e Libre de Bruxelles, Brussels, Belgium}
	
	\title{The \irace Package: User Guide}
	\date{Version \iraceversion, \today}
	%\keywords{automatic
		%  algorithm configuration, racing, parameter tuning, \aR}
	
	\maketitle
	
	\tableofcontents
	
	%Load files needed for examples
	<<exampleload,eval=TRUE,include=FALSE>>=
	library("irace")
	load("examples.Rdata")
	load("irace-acotsp.Rdata")
	load("log-ablation.Rdata")
	options(width = 70)
	@
	\newpage
	
	
	%%
	%%
	%%
	%% General info
	%%
	%%
	%%
	\section{General information}
	\MANUEL{Some things could be taken from the intro of the irace paper and reformulated.}
	\MANUEL{It would be good to mention that not only opt algorithms can be configured with irace, we say this in the paper.}
	
	\subsection{Background}
	\MANUEL{I would add a paragraph defining what is irace (a bit longer than the abstract above) and references to the literature so people can find more info. The first reference should be the irace TR.} \LESLIE{Here i guess we should say why tune an algorithm is a good idea, and why using irace is a better one.}
	The \irace package implements an \emph{iterated racing} procedure,
	which is an extension of Iterated F-race (\IFRACE)~\cite{BirYuaBal2010:emaoa}.  The main use of
	\irace is the automatic configuration of optimization and decision algorithms,
	that is, finding the most appropriate settings of an algorithm given a 
	set of instances of a problem. However, it may also be useful for 
	configuring other types of algorithms when performance depends on 
	the used parameter settings. It builds upon the \pkg{race} package 
	by Birattari and it is implemented in \aR. The \irace
	package is available from CRAN:
	%
	\begin{center}
		\url{https://cran.r-project.org/package=irace}
	\end{center}
	%
	More information about \irace is available at
	\url{http://iridia.ulb.ac.be/irace}.
	
	\subsection{Version}
	The current version of the \irace package is \iraceversion. Previous
	versions of the package can also be found in the \href{https://cran.r-project.org/package=irace}{CRAN website}.
	
	The algorithm underlying the current version of \irace and its motivation are
	described by \citet{LopDubPerStuBir2016irace}. The \textbf{adaptive capping
		mechanism} available from version $3.0$ is described by
	\citet{PerLopHooStu2017:lion}. Details of the implementation before version 2.0
	can be found in a previous technical report~\cite{LopDubStu2011irace}.
	%
	\begin{xwarningbox}
		Versions of \irace before 2.0 are not compatible with the file formats
		detailed in this document.
	\end{xwarningbox}
	
	
	\subsection{License}
	
	The \irace package is Copyright \copyright{} \the\year\ and distributed under the GNU General Public
	License version 3.0 (\url{http://www.gnu.org/licenses/gpl-3.0.en.html}).
	The \irace package is free software (software libre): You can redistribute it and/or modify it under the terms
	of the GNU General Public License as published by the Free Software Foundation,
	either version 3 of the License, or (at your option) any later version.
	
	The \irace package is distributed in the hope that it will be useful, but
	WITHOUT ANY WARRANTY; without even the implied warranty of MERCHANTABILITY or
	FITNESS FOR A PARTICULAR PURPOSE.
	
	Please be aware that the fact that this program is released as Free Software
	does not excuse you from scientific propriety, which obligates you to give
	appropriate credit! If you write a scientific paper describing research that
	made substantive use of this program, it is your obligation as a scientist to
	(a) mention the fashion in which this software was used in the Methods section;
	(b) mention the algorithm in the References section. The appropriate citation
	is:
	
	\begin{itemize}[leftmargin=3em]
		\item[] Manuel López-Ibáñez, Jérémie Dubois-Lacoste, Leslie Pérez Cáceres,
		Thomas Stützle, and Mauro Birattari. The \irace package: Iterated Racing for
		Automatic Algorithm Configuration. \emph{Operations Research Perspectives}, 3:43--58, 2016.
		doi:~\href{http://dx.doi.org/10.1016/j.orp.2016.09.002}{10.1016/j.orp.2016.09.002}
		
	\end{itemize}
	
	\section{Before starting}
	\MANUEL{I think this could be a bit more detailed by defining what is a parameter, a configuration, an instance, etc. but ok for now.}
	
	The \irace package provides an automatic configuration tool for
	tuning optimization algorithms, that is, automatically finding good configurations for the parameters values
	of a (target) algorithm saving the effort that normally requires manual
	tuning.
	
	\begin{figure}[t]
		\centering
		\includegraphics[width=0.6\textwidth]{irace-scheme}
		\caption{Scheme of \irace flow of information.}
		\label{fig:irace-scheme}
	\end{figure}
	
	Figure~\ref{fig:irace-scheme} gives a general scheme of how \irace works.
	\Irace receives as input a \emph{parameter space definition} corresponding to the
	parameters of the target algorithm that will be tuned, a set of \emph{instances}
	for which the parameters must be tuned for and a set of options for \irace that define the \emph{configuration scenario}. Then, \irace searches in the parameter search space for good performing algorithm
	configurations by executing the target algorithm on different instances and
	with different parameter configurations. A \parameter{targetRunner} must be provided to execute the target algorithm with a
	specific parameter configuration ($\theta$) and instance ($i$). The \parameter{targetRunner} function (or program) acts
	as an interface between the execution of the target algorithm and \irace: It
	receives the instance and configuration as arguments and must return the
	evaluation of the execution of the target algorithm.
	
	The following user guide contains guidelines for installing \irace, defining
	configuration scenarios, and using \irace to automatically configure your
	algorithms.
	
	%%
	%%
	%%
	%% Installation
	%%
	%%
	%%
	
	\section{Installation}
	
	\subsection{System requirements}
	\begin{itemize}
		\item \aR ($\text{version} \geq 3.2.0$) is required for running irace, but you don't
		need to know the \aR language to use it.  \aR is freely available and you can
		download it from the \aR project website
		(\url{https://www.r-project.org}). See \autoref{sec:installation} for a
		quick installation guide of \aR.
		
		\item For GNU/Linux and OS X, the command-line executable
		\code{parallel-irace} requires GNU Bash.  Individual examples may require additional software.
	\end{itemize}
	
	\subsection{\irace installation} \label{sec:irace install}
	
	The \irace package can be installed automatically within \aR or
	by manual download and installation. We advise to use the automatic
	installation unless particular circumstances do not allow it. The instructions
	to install \irace with the two mentioned methods are the following:
	
	\subsubsection[Install automatically within R]{Install automatically within \aR{}}
	
	Execute the following line in the \aR console to install the package:
	<<R_irace_install, prompt=FALSE, eval=FALSE>>=
	install.packages("irace")
	@
	Select a mirror close to your location, and test the installation in the \aR console with:
	<<R_irace_launch,eval=FALSE, prompt=FALSE>>=
	library("irace")
	q() # To exit R
	@
	
	Alternatively, within the \aR graphical interface, you may use
	the \code{Packages and data->Package installer} menu on OS X or the \code{Packages} menu on Windows.
	
	\subsubsection{Manual download and installation}
	
	From the \irace package CRAN website
	(\url{https://cran.r-project.org/package=irace}), download one of the three
	versions available depending on your operating system:
	\begin{itemize}
		\item \code{irace_\iraceversion.tar.gz} (Unix/BSD/GNU/Linux)
		\item \code{irace_\iraceversion.tgz} (OS X)
		\item \code{irace_\iraceversion.zip} (Windows)
	\end{itemize}
	
	To install the package on GNU/Linux and OS X, you must execute the following
	command at the shell (replace \code{<package>} with the path to the downloaded file, either \code{irace_\iraceversion.tar.gz} or \code{irace_\iraceversion.zip}):
	%
	\begin{lstlisting}[style=BashInputStyle]
		R CMD INSTALL <package>
	\end{lstlisting}
	
	To install the package on Windows, open \aR and execute the following line on
	the \aR console (replace \code{<package>} with the path to the downloaded file \code{irace_\iraceversion.zip}):
	%\LESLIE{Check that this actually works on internet says that  this: \code{Rscript -e "install.packages('foo.zip', repos = NULL)"} also works}
	%
	<<install_win1,eval=FALSE, prompt=FALSE>>=
	install.packages("<package>", repos = NULL)
	@
	
	If the previous installation instructions fail because of insufficient
	permissions and you do not have sufficient admin rights to install \irace
	system-wide, then you need to force a local installation.
	
	\subsubsection{Local installation}
	
	Let's assume you wish to install \irace on a path denoted by
	\code{<R_LIBS_USER>}, which is a filesystem path for which you have sufficient
	rights. This directory \textbf{must} exist before attempting the
	installation. Moreover, you must provide to \aR the path to this
	library when loading the package. However, the latter can be avoided by adding
	the path to the system variable \code{R_LIBS} or to the \aR internal variable
	\code{.libPaths}, as we will see below.\footnote{%
		On Windows, see also
		\url{https://cran.r-project.org/bin/windows/base/rw-FAQ.html\#I-don_0027t-have-permission-to-write-to-the-R_002d3_002e3_002e1_005clibrary-directory}.}
	
	
	On GNU/Linux or OS X, execute the following commands to install the
	package on a local directory:
	
	\begin{lstlisting}[style=BashInputStyle]
		export R_LIBS_USER="<R_LIBS_USER>"
		# Create R_LIBS_USER if it doesn't exist
		mkdir $R_LIBS_USER
		# Replace <package> with the path to the downloaded file.
		R CMD INSTALL --library=$R_LIBS_USER <package>
		# Tell R where to find R_LIBS_USER
		export R_LIBS=${R_LIBS_USER}:${R_LIBS}
	\end{lstlisting}
	
	On Windows, you can install the package on a local directory by executing the
	following lines in the \aR console:
	
	<<install_win2,eval=FALSE, prompt=FALSE>>=
	# Replace <package> with the path to the downloaded file.
	# Replace <R_LIBS_USER> with the path used for installation.
	install.packages("<package>", repos = NULL, lib = "<R_LIBS_USER>")
	# Tell R where to find R_LIBS_USER.
	# This must be executed for every new session.
	.libPaths(c("<R_LIBS_USER>", .libPaths()))
	@
	
	\subsubsection{Testing the installation and invoking irace}
	
	Once \irace has been installed, load the package and test that the installation
	was successful by opening an \aR console and executing:
	
	<<R_irace_test1, prompt=FALSE, eval=FALSE>>=
	# Load the package
	library("irace")
	# Obtain the installation path
	system.file(package = "irace")
	@
	
	The last command must print out the filesystem path  where \irace is
	installed.  In the remainder of this guide, the variable
	\code{\$IRACE_HOME} is used to denote this path. When executing any provided
	command that includes the \code{\$IRACE_HOME} variable do not forget to replace
	this variable with the installation path of \irace.
	
	On GNU/Linux or OS X, you can let the operating system know where to find \irace
	by defining the \code{\$IRACE_HOME} variable and adding it to the system
	\code{PATH}. Append the following commands to \path{~/.bash_profile}, \path{~/.bashrc} or
	\path{~/.profile}:
	%
	%<<linux_irace_path1,engine='bash',eval=FALSE>>=
	\begin{lstlisting}[style=BashInputStyle]
		# Replace <IRACE_HOME> with the irace installation path
		export IRACE_HOME=<IRACE_HOME>
		export PATH=${IRACE_HOME}/bin/:$PATH
		# Tell R where to find R_LIBS_USER
		# Use the following line only if local installation was forced
		export R_LIBS=${R_LIBS_USER}:${R_LIBS}
	\end{lstlisting}
	%@
	
	Then, open a new terminal and launch \irace as follows:
	
	%<<linux_irace_help,engine='bash',eval=FALSE>>=
	\begin{lstlisting}[style=BashInputStyle]
		irace --help
	\end{lstlisting}
	%@
	
	On Windows, you need to add both \aR and the installation path of \irace to the
	environment variable \code{PATH}. To edit the \code{PATH}, search for
	``Environment variables'' in the control panel, edit \code{PATH} and add
	a string similar to \path{C:\R_PATH\bin;C:\IRACE_HOME\bin\x64\} where \code{R_PATH}
		is the installation path of \aR and \code{IRACE_HOME} is the installation path
		of \irace. If \irace was installed locally, you also need to edit the
		environment variable \code{R_LIBS} to add \code{R_LIBS_USER}. Then, open a new
		terminal (run program \code{cmd.exe}) and launch \irace as:
		%
		%<<win_irace_help,engine='bash',eval=FALSE>>=
		\begin{lstlisting}[style=BashInputStyle]
			irace.exe --help
		\end{lstlisting}
		%@
		
		Alternatively, you may directly invoke \irace from within the \aR
		console by executing:
		<<windows_irace_help ,eval=FALSE, prompt=FALSE>>=
		library("irace")
		irace.cmdline("--help")
		@
		
		
		\section{Running irace}\label{sec:execution}
		
		Before performing the tuning of your algorithm, it is necessary to define a
		tuning scenario that will give \irace all the necessary information to optimize the
		parameters of the algorithm. The tuning scenario is composed of the following elements:
		
		\begin{enumerate}
			\item Target algorithm parameter description (see \autoref{sec:target parameters}).
			\item Target algorithm runner (see \autoref{sec:runner}).
			\item Training instances list (see \autoref{sec:training})
			\item \irace options (see \autoref{sec:irace options}).
			\item \textit{Optional:} Initial configurations (see \autoref{sec:initial}).
			\item \textit{Optional:} Forbidden configurations (see \autoref{sec:forbidden}).
			\item \textit{Optional:} Target algorithm evaluator (see \autoref{sec:evaluator}).
		\end{enumerate}
		
		These scenario elements can be provided as plain text files
		or as \aR objects. This user guide provides examples of both types,
		but we advise the use of plain text files, which we consider the simpler
		option.
		
		For a step-by-step guide to create the scenario elements for your target algorithm
		continue to \autoref{sec:step}. For an example execution of \irace using the
		\ACOTSP scenario go to \autoref{sec:example}.
		
		
		
		\subsection{Step-by-step setup guide}\label{sec:step}
		This section provides a guide to setup a basic execution of \irace. The
		template files provided in the package (\IRACEHOME{/templates}) will be
		used as basis for creating your new scenario. Please follow carefully the
		indications provided in each step and in the template files used; if you have
		doubts check the the sections that describe each option in detail.
		
		\begin{enumerate}[leftmargin=*]
			\item Create a directory (\eg~\path{~/tuning/}) for the scenario setup. This directory will contain
			all the files that describe the scenario. On GNU/Linux or OS X, you can do this as follows:
			%<<dir1,engine='bash',eval=FALSE>>=
			\begin{lstlisting}[style=BashInputStyle]
				mkdir ~/tuning
				cd ~/tuning
			\end{lstlisting}
			%@
			
			\item Copy all the template files from the \IRACEHOME{/templates/}
			directory to the scenario directory.
			%<<copy1,engine='bash',eval=FALSE>>=
			\begin{lstlisting}[style=BashInputStyle]
				# $IRACE_HOME is the installation directory of irace.
				cp $IRACE_HOME/templates/*.tmpl  ~/tuning/
			\end{lstlisting}
			%@
			
			% Remember that \IRACEHOME{} is the path to the installation
			% directory of \irace. It can be obtained in the \aR console with:
			%
			% <<irace_path,eval=FALSE, prompt=FALSE>>=
			% library("irace")
			% system.file(package = "irace")
			% @
			
			\item For each template in your tuning directory, remove the \code{.tmpl}
			suffix, and modify them following the next steps.
			
			\item Define the target algorithm parameters to be tuned by following the
			instructions in \code{parameters.txt}.  Available parameter types and other
			guidelines can be found in \autoref{sec:target parameters}.
			
			
			\item \textit{Optional}: Define the initial parameter configuration(s) of your
			algorithm, which  allows you to provide good starting configurations (if you know
			some) for the tuning. Follow the instructions in \code{configurations.txt} and set \parameter{configurationsFile}\code{="configurations.txt"} in \code{scenario.txt}. More
			information in \autoref{sec:initial}. If you do not need to
			define initial configurations remove this file from the directory.
			
			\item \textit{Optional}: Define forbidden parameter value combinations,
			that is, configurations that \irace must not consider in the tuning.
			Follow the instructions in \code{forbidden.txt} and update \code{scenario.txt} with \parameter{forbiddenFile} \code{=} \code{"forbidden.txt"}. More information about
			forbidden configurations in \autoref{sec:forbidden}. 
			If you do not need to define forbidden configurations remove this file from
			the directory.
			
			\item Place the instances you would like to use for the tuning of your
			algorithm in the folder \path{~/tuning/Instances/}. In addition, you can
			create a file (\eg~\code{instances-list.txt}) that specifies which instances
			from that directory should be run and which instance-specific parameters to
			use. To use such an instance file, set the appropriate option in
			\code{scenario.txt}, e.g., \parameter{trainInstancesFile} \code{= "instances-list.txt"}.
			See \autoref{sec:training} for guidelines.
			
			\item Uncomment and assign in \code{scenario.txt} only the options for which you
			need a value different from the default.
			%% MANUEL: I'm not sure what this means.
			% The names of the template files match the default names of the scenario
			% options.
			Some common parameters that you might want to adjust are:
			\begin{description}
				\item[\parameter{execDir}] (\code{--exec-dir}): the directory in which \irace will execute the
				target algorithm; the default value is the current directory.
				%\item[\parameter{logFile}] (\code{--log-file}): a file the name of the results \aR data file that produces \irace.
				\item[\parameter{maxExperiments}] (\code{--max-experiments}): the maximum number of executions
				of the target algorithm that \irace will perform.
				\item[\parameter{maxTime}] (\code{--max-time}): maximum total execution time in seconds for the executions of \code{targetRunner}. In this case, \code{targetRunner} must return two values: cost and time. Note that you must provide either \parameter{maxTime} or \parameter{maxExperiments}.
			\end{description}
			For setting the tuning budget see \autoref{sec:budget}. For more information on \irace options and their default values, see \autoref{sec:irace options}.
			
			\item Modify the \code{target-runner} script to run your algorithm. This script
			must execute your algorithm with the parameters and instance specified by
			\irace and return the evaluation of the execution and \textit{optionally} the execution 
			time (\code{cost [time]}). When the \parameter{maxTime} option is used, returning \code{time} is mandatory.
			The \code{target-runner} template is written in \proglang{GNU Bash}
			scripting language, which can be executed easily in GNU/Linux and OS X
			systems. However, you may use any other programming language. As an example,
			we provide a \proglang{Python} example in the directory
			\IRACEHOME{/examples/python}. %
			Follow these instructions to adjust the given \code{target-runner} template
			to your algorithm:
			\begin{enumerate}
				\item Set the \code{EXE} variable with the path to the executable of the target algorithm.
				\item Set the \code{FIXED_PARAMS} if you need extra arguments in the
				execution line of your algorithm. An example could be the time that your
				algorithm is required to run
				(\code{FIXED_PARAMS}\hspace{0pt}\code{=}\hspace{0pt}\code{"--time 60"}) or
				the number of evaluations required (\code{FIXED_PARAMS}\hspace{0pt}\code{=}\hspace{0pt}\code{"--evaluations 10000"}).
				
				\item The line provided in the template executes the executable described in the \code{EXE} variable.
				%
				\begin{center}
					\code{\$EXE \$\{FIXED_PARAMS\} -i \$\{INSTANCE\} --seed \$\{SEED\} \$\{CONFIG_PARAMS\}}
				\end{center}
				%
				You must change this line according to the way your algorithm is executed. In this
				example, the algorithm receives the instance to solve with the flag
				\code{-i} and the seed of the random number generator with the flag \code{--seed}. The variable
				\code{CONFIG_PARAMS} adds to the command line the parameters that \irace has
				given for the execution.  You must set the command line execution as
				needed. For example, the instance might not need a flag and might need to be
				the first argument:
				\begin{center}
					\code{\$EXE \$\{INSTANCE\} \$\{FIXED_PARAMS\} --seed \$\{SEED\} \$\{CONFIG_PARAMS\}}
				\end{center}
				The output of your algorithm is saved to the file defined in the
				\code{\$STDOUT} variable, and error output is saved in the file given by
				\code{\$STDERR}. The line:
				\begin{center}
					\code{if [ -s "\${STDOUT}" ]; then}
				\end{center}
				checks if the file containing the output of your algorithm is not
				empty. The example provided in the template assumes that your algorithm
				prints in the last output line the best result found (only a number). The
				line:
				\begin{center}
					\code{COST=\$(cat \$\{STDOUT\} | grep -e '\^{}[[:space:]]*[+-]\textbackslash{}?[0-9]' | cut -f1)}
				\end{center}
				parses the output of your algorithm to obtain the result from the last line. The \code{target-runner}
				script must return \textbf{only} one number. In the template example, the result is returned with
				\code{echo "\$COST"} (assuming \parameter{maxExperiments} is used) and the used files are deleted.
				
				\begin{xwarningbox}
					The \code{target-runner} script must be executable.
				\end{xwarningbox}
				
				You can test the target runner from the \aR console by checking the scenario as 
				explained earlier in \autoref{sec:execution}.
				
				If you have problems related to the \code{target-runner} script when executing
				\irace, see \autoref{sec:check list} for a check list to help diagnose common
				problems. For more information about the \parameter{targetRunner}, please see
				\autoref{sec:runner},
			\end{enumerate}
			
			\item \textit{Optional}: Modify the \code{target-evaluator} file. This is rarely needed and the \code{target-runner} template does not use it. \autoref{sec:evaluator} explains when a \parameter{targetEvaluator} is needed and how to define it.
			
			\item The \irace executable provides an option (\parameter{-{}-check}) to
			check that the scenario is correctly defined. We recommend to perform a
			check every time you create a new scenario. When performing the check,
			\irace will verify that the scenario and parameter definitions are
			correct and will test the execution of the target algorithm. To check
			your scenario execute the following commands:
			
			\begin{itemize}
				\item From the command-line (on Windows, execute \code{irace.bat}):
				
				%<<irace_check,engine='bash', eval=FALSE>>=
				\begin{lstlisting}[style=BashInputStyle]
					# $IRACE_HOME is the installation directory of irace.
					$IRACE_HOME/bin/irace --scenario scenario.txt --check
				\end{lstlisting}
				%@
				
				\item Or from the \aR console:
				
				<<irace_R_check, eval=FALSE, prompt=FALSE>>=
				library("irace")
				scenario <- readScenario(filename = "scenario.txt",
				scenario = defaultScenario())
				checkIraceScenario(scenario = scenario)
				@
				
			\end{itemize}
			
			
			\item  Once all the scenario elements are prepared you can execute \irace, either using the command-line wrappers provided by the package
			or directly from the \aR console:
			
			\begin{itemize}
				\item{%
					\textbf{From the command-line console}, call the command (on Windows, you should execute
					\code{irace.exe}):
					\begin{lstlisting}[style=BashInputStyle]
						cd ~/tuning/
						# $IRACE_HOME is the installation directory of irace
						# By default, irace reads scenario.txt, you can specify a different file
						# with --scenario.
						$IRACE_HOME/bin/irace
					\end{lstlisting}
					For this example we assume that the needed scenario files have been set
					properly in the \code{scenario.txt} file using the options described in
					\autoref{sec:irace options}. Most \irace options can be specified
					in the command line or directly in the \code{scenario.txt} file.
				}
				\item{
					\textbf{From the \aR console}, evaluate:
					
					<<irace_R_exe, eval=FALSE, prompt=FALSE>>=
					library("irace")
					# Go to the directory containing the scenario files
					setwd("~/tuning")
					scenario <- readScenario(filename = "scenario.txt",
					scenario = defaultScenario())
					irace.main(scenario = scenario)
					@
				}
			\end{itemize}
			
			This will perform one run of \irace. See the output of \code{irace --help} in the command-line or \code{irace.usage()} in \aR for quick information on
			additional \irace parameters. For more information about \irace options, see \autoref{sec:irace options}.
			
		\end{enumerate}
		
		\begin{xwarningbox}
			Command-line options override the same options specified in the \code{scenario.txt} file.
		\end{xwarningbox}
		
		\subsection{Setup example for ACOTSP}\label{sec:example}
		
		The \ACOTSP tuning example can be found in the package installation at 
		\IRACEHOME{/examples/acotsp}.
		%
		Additionally, a number of example scenarios can be found in the \code{examples} folder. More
		examples of tuning scenarios can be found in the Algorithm Configuration Library (AClib, \url{http://www.aclib.net/}).
		
		In this section, we describe how to execute the \ACOTSP scenario. If you wish to start setting up
		your own scenario, continue to the next section. For this example, we assume
		a GNU/Linux system but making the necessary changes in the commands and \parameter{targetRunner},
		it can be executed in any system that has a \proglang{C} compiler.
		%\MANUEL{I don't think this is true, since the target-runner script needs Bash}
		To execute this scenario follow these steps:
		
		\begin{enumerate}
			\item Create a directory for the tuning (\eg~\path{~/tuning/}) and copy the example scenario files
			located in the \code{examples} folder to the created directory:
			%<<dir0,engine='bash',eval=FALSE>>=
			\begin{lstlisting}[style=BashInputStyle]
				mkdir ~/tuning
				cd ~/tuning
				# $IRACE_HOME is the installation directory of irace.
				cp $IRACE_HOME/examples/acotsp/* ~/tuning/
			\end{lstlisting}
			%@
			
			\item Download the training instances from \url{http://iridia.ulb.ac.be/irace/} to the \path{~/tuning/} directory.
			\item Create the instance directory (\eg~\path{~/tuning/Instances}) and decompress the instance files on it.
			
			%<<instance0,engine='bash',eval=FALSE>>=
			\begin{lstlisting}[style=BashInputStyle]
				mkdir ~/tuning/Instances/
				cd ~/tuning/
				tar -xvf tsp-instances-training.tar.bz2 Instances/
			\end{lstlisting}
			%@
			
			\item Download the \ACOTSP software from \url{http://www.aco-metaheuristic.org/aco-code/} to the \path{~/tuning/}
			directory and compile it.
			%<<acotsp0,engine='bash',eval=FALSE>>=
			\begin{lstlisting}[style=BashInputStyle]
				cd ~/tuning/
				tar -xvf ACOTSP-1.03.tgz
				cd ~/tuning/ACOTSP-1.03
				make
			\end{lstlisting}
			%@
			\item Create a directory for the executable and copy it:
			
			%<<acotsp1,engine='bash',eval=FALSE>>=
			\begin{lstlisting}[style=BashInputStyle]
				mkdir ~/bin/
				cp ~/tuning/ACOTSP-1.03/acotsp ~/bin/
			\end{lstlisting}
			%@
			
			\item Create a directory for executing the experiments and execute \irace:
			
			%<<runexample,engine='bash',eval=FALSE>>=
			\begin{lstlisting}[style=BashInputStyle]
				mkdir ~/tuning/acotsp-arena/
				cd ~/tuning/
				# $IRACE_HOME is the installation directory of irace.
				$IRACE_HOME/bin/irace
			\end{lstlisting}
			%@
			
			\item Or you can also execute \irace from the \aR console using:
			
			<<runexample2,prompt=FALSE,eval=FALSE>>=
			library("irace")
			setwd("~/tuning/")
			irace.cmdline()
			@
			
		\end{enumerate}
		
		%%
		%%
		%%
		%% Scenario settings
		%%
		%%
		%%
		
		
		
		\section{Defining a configuration scenario}\label{sec:scenario}
		\subsection{Target algorithm parameters} \label{sec:target parameters}
		
		The parameters of the target algorithm are defined by a parameter file as
		described in \autoref{sec:parameters file}. Optionally, when executing
		\irace from the \aR console, the parameters can be specified directly as an \aR
		object (see \autoref{sec:parameters object}). For defining your parameters
		follow the guidelines provided in the following sections.
		
		\subsubsection{Parameter types}
		
		Each target parameter has an associated type that defines its domain and the
		way \irace handles them internally. Understanding the nature of the domains of
		the target parameters is important to select appropriate types. The four basic
		types supported by \irace are the following:
		
		\begin{itemize}
			\item \textit{Real} parameters are numerical parameters that can take
			floating-point values within a given range. The range is
			specified as an interval `\code{(<lower bound>,<upper
				bound>)}'. This interval is closed, that is, the parameter value
			may eventually be one of the bounds. The possible values are rounded
			to a number of \emph{decimal places} specified by
			option \parameter{digits}. For example, given the default
			number of digits of $4$, the values $0.12345$ and
			$0.12341$ are both rounded to $0.1234$.
			% However, the values  $0.00001$ and $0.00005$ remain the same.
			Selected real-valued parameters can be optionally sampled on
			a logarithmic scale (base $e$).
			
			\item \textit{Integer} parameters are numerical parameters that can take only
			integer values within the given range. Their range is specified as the range
			of real parameters and they can also be optionally sampled on a
			logarithmic scale (base $e$).
			
			\item \textit{Categorical} parameters are defined by a set of possible values
			specified as `\code{(<value 1>,} \code{...,} \code{<value n>)}'. The values
			are quoted or unquoted character strings. Empty strings and strings
			containing commas or spaces must be quoted.
			
			\item \emph{Ordinal} parameters are defined by an \emph{ordered} set
			of possible values in the same format as for categorical
			parameters. They are handled internally as integer parameters, where
			the integers correspond to the indexes of the values.
			
		\end{itemize}
		
		\subsubsection{Parameter domains}
		
		For each target parameter, an interval or a set of values must be defined
		according to its type, as described above. There is no limit for the size of
		the set or the length of the interval, but keep in mind that larger domains
		could increase the difficulty of the tuning task. Choose always values that you
		consider relevant for the tuning. In case of doubt, we recommend to choose larger 
		intervals, as occasionally best parameter settings may be not intuitive a priori. 
		All intervals are considered as closed intervals.
		
		It is possible to define parameters that will have always the same value.
		Such ``\emph{fixed}'' parameters will not be tuned but their values are used
		when executing the target algorithm and they are affected by constraints
		defined on them. All fixed parameters must be defined as categorical
		parameters and have a domain of one element.
		
		\subsubsection{Conditional parameters}
		
		Conditional parameters are active only when others have certain values. These
		dependencies define a hierarchical relation between parameters. For example,
		the target algorithm may have a parameter \code{localsearch} that takes values
		\code{(sa, ts)} and another parameter \code{ts-length} that only needs to be
		set if the first parameter takes precisely the value \code{ts}. Thus, parameter
		\code{ts-length} is conditional on \code{localsearch == "ts"}.
		
		\subsubsection{Parameter file format}\label{sec:parameters file}
		
		For simplicity, the description of the parameters space is given as a
		table. Each line of the table defines a configurable parameter
		%
		\begin{center}
			\code{<name>\ <label>\ <type>\ <range>\ [ | <condition>\ ] }
		\end{center}
		where each field is defined as follows:
		%
		\begin{center}
			\renewcommand{\arraystretch}{1.2}
			\begin{tabularx}{0.98\linewidth}{@{}rX@{}}
				\code{<name>} & The name of the parameter as an unquoted
				alphanumeric string, e.g., `\code{ants}'.\\
				%
				\code{<label>}& A \emph{label} for this parameter. This is a
				string that will be passed together with the parameter
				to \parameter{targetRunner}. In the default \parameter{targetRunner}
				provided with the package (\autoref{sec:runner}), this is the
				command-line switch used to pass the value of this parameter, for
				instance `\code{"-{}-ants "}'.\newline
				The value of the parameter is concatenated \emph{without
					separator} to the label when
				invoking \parameter{targetRunner}, thus \emph{any whitespace in the label is significant}. Following the same example, when parameter \code{ants} takes value \code{5}, the default targetRunner will pass the parameter as \code{"-{}-ants 5"}.\\
				%
				\code & The type of the parameter, either
				\textit{integer}, \textit{real}, \textit{ordinal} or
				\textit{categorical}, given as a single letter: `\code{i}',
				`\code{r}', `\code{o}' or `\code{c}'. Numerical parameters
				can be sampled using a natural logarithmic scale with '\code{i,log}' and
				'\code{r,log}' (without spaces) for integer and real
				parameters, respectively.\\
				%
				\code & The range or set of values of the parameter delimited by
				parentheses. \eg~\code{(0,1)} or \code{(a,b,c,d)}.\\
				%
				
				\code & An optional \emph{condition} that determines whether the parameter is
				enabled or disabled, thus making the parameter conditional. If the
				condition evaluates to false, then no value is assigned to this
				parameter, and neither the parameter value nor the corresponding
				label are passed to \parameter{targetRunner}. The condition must follow the same syntax as those for specifying forbidden configurations (\autoref{sec:forbidden}), that is, it must be a
				valid \aR logical expression\footnote{For a list of \aR operators
					see: \url{https://stat.ethz.ch/R-manual/R-devel/library/base/html/Syntax.html}}. 
				The condition may contain the name
				of other parameters as long as the dependency graph does not
				contain any cycle. Otherwise, \irace will detect the cycle and
				stop with an error. \\
			\end{tabularx}
		\end{center}
		
		As an example, Figure~\ref{fig:acotsp_parameters} shows the parameters file
		of the \ACOTSP scenario.
		
		\begin{figure}[!hbt]
			\small\centering%
			\begin{CodeInput}
				# name      switch            type     values               [conditions (using R syntax)]
				algorithm   "--"              c        (as,mmas,eas,ras,acs)
				localsearch "--localsearch "  c        (0, 1, 2, 3)
				alpha       "--alpha "        r        (0.00, 5.00)
				beta        "--beta "         r        (0.00, 10.00)
				rho         "--rho  "         r        (0.01, 1.00)
				ants        "--ants "         i        (5, 100)
				nnls        "--nnls "         i        (5, 50)              | localsearch %in% c(1, 2, 3)
				q0          "--q0 "           r        (0.0, 1.0)           | algorithm == "acs"
				dlb         "--dlb "          c        (0, 1)               | localsearch %in% c(1,2,3)
				rasrank     "--rasranks "     i        (1, 100)             | algorithm == "ras"
				elitistants "--elitistants "  i        (1, 750)             | algorithm == "eas"
			\end{CodeInput}
			\caption{Parameter file (\code{parameters.txt}) for tuning \ACOTSP.}\label{fig:acotsp_parameters}
		\end{figure}
		
		\subsubsection{Parameters R format}\label{sec:parameters object}
		The target parameters are stored in an \aR~\code{list} that you can obtain from the \aR console using the following command:
		
		<<readParameters,prompt=FALSE, eval=FALSE>>=
		parameters <- readParameters(file = "parameters.txt")
		@
		
		See the help of the \code{readParameters} function (\code{?readParameters}) for more information.
		The structure of the parameter list that is created is as follows:
		
		\begin{center}
			\renewcommand{\arraystretch}{1.2}
			\begin{tabularx}{0.98\linewidth}{@{}rX}
				\code{names}        & Vector that contains the names of the parameters. \\
				\code{types}        & Vector that contains the type of each parameter 'i', 'c', 'r', 'o'. \\
				\code{switches}     & Vector that contains the labels of the parameters. e.g., switches to be used for the
				parameters on the command line. \\
				\code{domain}       & List of vectors, where each vector may contain two
				values (minimum, maximum) for real and integer parameters, or
				a set of values for categorical and ordinal parameters.\\
				\code{conditions}   & List of \aR logical expressions, with variables
				corresponding to parameter names.\\
				\code{isFixed}      & Logical vector that specifies which parameter is fixed
				and, thus, it does not need to be tuned.\\
				\code{transform}    & Vector that contains the transformation of each parameter. Currently, it
				can take values \code{``''} (no transformation, default) of \code{``log''}
				(natural logarithmic transformation).\\
				\code{nbParameters} & An integer, the total number of parameters.\\
				\code{nbFixed}      & An integer, the number of parameters with a fixed value.\\
				\code{nbVariable}   & Number of variable (i.e., to be tuned) parameters.\\
			\end{tabularx}
		\end{center}
		
		The following example shows the structure of the \code{parameters} \aR object
		for the \code{algorithm}, \code{ants} and \code{q0} parameters of the \ACOTSP scenario:
		
		<<parametersetup,eval=TRUE,include=FALSE>>=
		# Setup example
		parameters <- irace::readParameters(text='
		algorithm       "--"                 c		(as,mmas,eas,ras,acs)
		ants            "--ants "            i	  	(5, 100)
		q0              "--q0 "              r  	(0.0, 1.0) 		| algorithm %in% c("acs")
		')
		@
		
		<<parameterlist,eval=TRUE, prompt=TRUE, size='normalsize', comment="">>=
		str(parameters, vec.len = 10)
		@
		
		\subsection{Target algorithm runner}
		\label{sec:runner}
		
		The evaluation of a candidate configuration on a single instance is done by means of a user-given auxiliary
		program or, alternatively, a user-given \aR function. The function (or program name) is specified by the option
		\parameter{targetRunner}. The \parameter{targetRunner} must return the
		cost value (\eg cost of the best solution found) of the evaluation; unless computing the cost requires information from all the configurations evaluated on an instance, \eg when evaluating multi-objective algorithms with unknown normalisation bounds (see \autoref{sec:evaluator} for details).
		
		\begin{xwarningbox}
			The objective of \irace is to minimize the cost value returned by the target
			algorithm.  If you wish to maximize, you can multiply the cost by
			\code{-1} before returning it to \irace.
		\end{xwarningbox}
		
		\subsubsection{Target runner executable program}
		
		When \parameter{targetRunner} is an auxiliary executable program, it
		is invoked for each candidate configuration, passing as arguments:
		%
		\begin{center}
			\small\code{<id.configuration>\ <id.instance>\ <seed>\ <instance>\ [bound]\ <configuration>}
		\end{center}
		%
		\begin{center}
			\renewcommand{\arraystretch}{1.2}
			\begin{tabularx}{1\linewidth}{@{}rX}
				\code{id.configuration}  & an alphanumeric string that uniquely identifies a configuration;\\
				\code{id.instance} & an alphanumeric string that uniquely identifies an instance;\\
				\code{seed} & seed for the random number generator to be used for this evaluation, ignore the seed for deterministic algorithms;\\
				\code{instance} & string giving the instance to be used for this evaluation;\\
				\code{bound} & optional execution time bound. Only provided when the \parameter{maxBound} option is set in the scenario, see \autoref{sec:capping};\\
				\code{configuration} & the pairs parameter label-value that
				describe this candidate configuration. Typically given as command-line
				switches to be passed to the executable program.\\
			\end{tabularx}
		\end{center}
		
		The experiment list shown in \autoref{sec:runner fx}, would result in the following execution line:
		
		%<<target_runner_ex1, engine='bash', eval=FALSE>>=
		\begin{lstlisting}[style=BashInputStyle]
			target-runner 1 113 734718556 /home/user/instances/tsp/2000-533.tsp \
			--eas --localsearch 0 --alpha 2.92 --beta 3.06 --rho 0.6 --ants 80
		\end{lstlisting}
		%@
		
		The command line switches that describe the candidate configuration are
		constructed by appending to each parameter label (switch), \emph{without
			separator}, the value of the parameter, following the order given in the
		parameter table. The program \parameter{targetRunner} must print a real
		number, which corresponds to the cost measure of the candidate configuration
		for the given instance and optionally its execution time (mandatory when \parameter{maxTime} 
		is used and/or when the \parameter{capping} option is enabled). The working directory of 
		\parameter{targetRunner} is set to the execution directory specified by the option
		\parameter{execDir}. This allows the user to execute independent runs of \irace
		in parallel using different values for \parameter{execDir}, without the runs
		interfering with each other.
		
		\subsubsection{Target runner R function}
		\label{sec:runner fx}
		
		When \parameter{targetRunner} is an \aR function, it is invoked
		for each candidate configuration as:
		%
		<<targetRunner,prompt=FALSE, eval=FALSE>>=
		targetRunner(experiment, scenario)
		@
		%
		\noindent where \code{experiment} is a list that contains  information about
		configuration and instance to execute one experiment, and \code{scenario} is the scenario
		list. The structure of the \code{experiment} list is as follows:
		
		\begin{center}
			\renewcommand{\arraystretch}{1.2}
			\begin{tabularx}{1\linewidth}{@{}rX}
				\code{id.configuration}  & an alphanumeric string that uniquely identifies a configuration;\\
				\code{id.instance} & an alphanumeric string that uniquely identifies an instance;\\
				\code{seed} & seed to be used for this evaluation;\\
				\code{instance} & string giving the instance to be used for this evaluation;\\
				\code{bound} &  optional execution time bound;\\
				\code{configuration}    & 1-row data frame with a column per parameter name;\\
				\code{switches}         & vector of parameter switches (labels) in the order of parameters used in \code{configuration}.\\
				
			\end{tabularx}
		\end{center}
		
		The following is an example of an experiment list for the \ACOTSP scenario:
		
		<<experimentlist,eval=TRUE,size='normalsize', prompt=TRUE,  comment="">>=
		print(experiment)
		@
		
		If \parameter{targetEvaluator} is \code{NULL}, then
		the \parameter{targetRunner} function must return a list with at least one
		element \code{"cost"}, the numerical value corresponding to the evaluation of
		the given configuration on the given instance. A cost of \code{Inf} is accepted
		and results in the immediate rejection of the configuration (see
		\autoref{sec:reject}).
		
		If the scenario option \parameter{maxTime} is non-zero or if the \parameter{capping} 
		option is enabled, then the list must contain at least another element \code{"time"} that 
		reports the execution time for this call to \code{targetRunner}.
		
		The return list may also contain the following optional elements that are used
		by \irace for reporting errors in \code{targetRunner}:
		%
		\begin{center}
			\renewcommand{\arraystretch}{1.2}
			\begin{tabularx}{1\linewidth}{@{}rX}
				\code{error} & is a string used to report an error;\\
				\code{outputRaw} & is a string used to report the raw output of calls to
				an external program or function;\\
				\code{call} & is a string used to report how \code{targetRunner} called 
				an external program or function;\\
			\end{tabularx}
		\end{center}
		
		
		
		\subsection{Target evaluator} \label{sec:evaluator}
		
		Normally, \parameter{targetRunner} returns the cost of the execution of a
		candidate configuration (see \autoref{sec:runner}). However, there are cases
		when the cost evaluation must be delayed until all candidate configurations in
		a race have been executed on a instance.
		
		The \parameter{targetEvaluator} option defines an auxiliary program (or an \aR
		function) that allows postponing the evaluations of the candidate
		configurations. For each instance seen, the program \parameter{targetEvaluator} is only invoked after all
		the calls to \parameter{targetRunner} for all alive candidate configurations on
		the same instance have already finished.
		
		\begin{xwarningbox}
			When using \parameter{targetEvaluator}, \parameter{targetRunner} must not return the evaluation
			of the configuration. If \parameter{maxTime} is used, \parameter{targetRunner} must return only 
			execution time.
		\end{xwarningbox}
		
		As an example, \parameter{targetEvaluator} may be used to dynamically find
		normalization bounds for the output returned by an algorithm for each
		individual instance. In this case, \parameter{targetRunner} will save the
		output of the algorithm, then the first call to \parameter{targetEvaluator}
		will examine the output produced by all calls to \parameter{targetRunner} for
		the same instance, update the normalization bounds and return the normalized
		output. Subsequent calls to \parameter{targetEvaluator} for the same instance
		will simply return the normalized output.
		
		A similar need arises when using quality measures for multi-objective
		optimization algorithms, such as the hypervolume, which typically
		require specifying reference points or sets. By
		using \parameter{targetEvaluator}, it is possible to dynamically compute the
		reference points or sets while \irace is running. Examples are provided at
		\path{examples/hypervolume}. See also \autoref{sec:multi objective} for
		more information on how to tune multi-objective algorithms.
		
		\subsubsection{Target evaluator executable program}
		
		When \parameter{targetEvaluator} is an auxiliary executable program, it
		is invoked for each candidate with the following arguments:
		%
		\begin{center}
			\small\code{<id.configuration>\ <id.instance>\ <seed>\ <instance>\ <num.configs>\ <all.conf.id>}
		\end{center}
		%
		\begin{center}
			\renewcommand{\arraystretch}{1.2}
			\begin{tabularx}{1\linewidth}{@{}rX}
				\code{id.configuration}  & an alphanumeric string that uniquely identifies a configuration;\\
				\code{id.instance} & an alphanumeric string that uniquely identifies an instance;\\
				\code{seed} & seed to be used for this evaluation;\\
				\code{instance} & string giving the instance to be used for this evaluation;\\
				\code{num.configs} & number of alive candidate configurations;\\
				\code{all.conf.id}    & list of IDs of the alive configurations separated by whitespace.\\
			\end{tabularx}
		\end{center}
		
		The \parameter{targetEvaluator} executable must print a numerical value
		corresponding to the cost measure of the candidate configuration on
		the given instance.
		
		
		
		
		\subsubsection{Target evaluator R function}
		
		When \parameter{targetEvaluator} is an \aR function, it is invoked
		for each candidate configuration as:%
		%
		<<targetEvaluator, prompt=FALSE, eval=FALSE>>=
		targetEvaluator(experiment, num.configurations, all.conf.id,
		scenario, target.runner.call)
		@
		%
		\noindent where \code{experiment} is a list that contains information about one
		experiment (see \autoref{sec:runner fx}), \code{num.configurations}
		is the number of configurations alive in the race, \code{all.conf.id} is
		the vector of IDs of the alive configurations,  \code{scenario} is
		the scenario list and \code{target.tunner.call} is the string of the
		\parameter{targetRunner} execution line.
		
		The function \parameter{targetEvaluator} must return a list with one element
		\code{"cost"}, the numerical value corresponding to the cost measure of the
		given configuration on the given instance.
		
		The return list may also contain the following optional elements that are used
		by \irace for reporting errors in \code{targetEvaluator}:
		%
		\begin{center}
			\renewcommand{\arraystretch}{1.2}
			\begin{tabularx}{1\linewidth}{@{}rX}
				\code{error} & is a string used to report an error;\\
				\code{outputRaw} & is a string used to report the raw output of calls to
				an external program or function;\\
				\code{call} & is a string used to report how \code{targetEvaluator} called 
				an external program or function;\\
			\end{tabularx}
		\end{center}
		
		\subsection{Training instances}\label{sec:training}
		
		The \irace options \parameter{trainInstancesDir} and \parameter{trainInstancesFile}
		specify where to find the training instances.
		%
		By default, the value of \parameter{trainInstancesFile} is empty. This means that \irace will consider all files within the directory given by \parameter{trainInstancesDir} (by default \path{./Instances}) as training instances. 
		
		Otherwise, the value of \parameter{trainInstancesFile} may specify a text
		file. The format of this file is one instance per line. Within each line, elements separated by white-space will be parsed as separate arguments to be supplied to \parameter{targetRunner}. This allows defining instance-specific parameter settings. Quoted strings will be parsed as a single argument.
		%
		The following example shows a training instance file for the \ACOTSP scenario:
		
		\begin{figure}[!ht]
			\centering%
			\begin{minipage}{0.4\textwidth}
				\begin{CodeInput}
					# Example training instances file
					100/100-1_100-2.tsp --time 1
					100/100-1_100-3.tsp --time 2
					100/100-1_100-4.tsp --time 3
				\end{CodeInput}
			\end{minipage}
			\caption{Training instances file for tuning \ACOTSP.}\label{fig:acotsp_training}
		\end{figure}
		
		The value of \parameter{trainInstancesDir} is always prefixed to the instance
		name, that is, the instances names are treated as relative to this directory.
		%
		For example, given the above file as \parameter{trainInstancesFile} and the default value of  \parameter{trainInstancesDir} (\code{./Instances}), then a possible invocation of \parameter{targetRunner} would be:
		%
		%<<target_runner_cmd, engine='bash', eval=FALSE>>=
		\begin{lstlisting}[style=BashInputStyle]
			target-runner 1 113 734718 ./Instances/100/100-1_100-2.tsp --time 1 \
			--alpha 2.92 ...
		\end{lstlisting}
		%@
		
		
		Training instances do not need to be files, \irace just passes the elements of each line as arguments to \parameter{targetRunner}, thus each line may denote the name of a benchmark function or a label, plus instance-specific settings, that the target algorithm understands. Each line may even be the command-line parameters required to call an instance generator within \parameter{targetRunner}. When the instances do not represent actual files, then \parameter{trainInstancesDir} is usually set to the empty string (\code{--train-instances-dir=""}). For example,
		
		\begin{center}
			\begin{minipage}{0.4\linewidth}
				\begin{CodeInput}
					# Example training instances file
					rosenbrock_20 --function=12 --nvar 20
					rosenbrock_30 --function=12 --nvar 30
					rastrigin_20 --function=15 --nvar 20
					rastrigin_30 --function=15 --nvar 30
				\end{CodeInput}
			\end{minipage}
		\end{center}
		%\caption{Example of training instances file where instances do not correspond to files.}\label{fig:nofile_training}
		%\end{figure}
		
		Optionally, when executing \irace from the \aR console, the list of instances might be provided explicitly by means of the variable \code{scenario$instances}. Thus, the previous example would be equivalent to:
		
		<<instance1, prompt=FALSE, eval=FALSE>>=
		scenario$instances <- c("rosenbrock_20 --function=12 --nvar 20",
		"rosenbrock_40 --function=12 --nvar 30",
		"rastrigin_20 --function=15 --nvar 20",
		"rastrigin_40 --function=15 --nvar 30")
		@
		
		By default, \irace assumes that the target algorithm is stochastic (the value
		of the option \parameter{deterministic} is $0$), thus, the same configuration
		can be executed more than once on the same instance and obtain different
		results.  In this case, \irace generates pairs \code{(instance, seed)} by
		generating a random seed for each instance. In other words, configurations
		evaluated on the same instance use the same random seed. This is a well-known
		variance reduction technique called \emph{common random
			numbers}~\citep{McG1992vrt}. If all available pairs are used within a run of
		\irace, new pairs are generated with different seeds, that is, a configuration
		evaluated more than once per instance will use different random seeds.
		
		If \parameter{deterministic} is set to $1$, then each instance will be used at
		most once per race. This setting should only be used for target algorithms that
		do not have a stochastic behavior and, therefore, executing the target
		algorithm on the same instance several times with different seeds does not make
		sense.
		\begin{xwarningbox}
			If \parameter{deterministic} is active and the number of training instances
			provided to \irace is less than \parameter{firstTest} (default: 5), no
			statistical test will be performed on the race.
		\end{xwarningbox}
		
		Finally, \irace randomly re-orders the sequence of instances provided. This
		random sampling may be disabled by using the option \parameter{sampleInstances}
		(\code{--sample-instances 0}) if keeping the order provided in the instance
		file is important.
		
		\begin{xwarningbox}
			We advise to always sample instances to prevent biasing the tuning due to the
			instance order. See also \autoref{sec:het}
		\end{xwarningbox}
		
		
		
		\subsection{Initial configurations} \label{sec:initial}
		
		The scenario option \parameter{configurationsFile} allows specifying a text
		file that contains an initial set of configurations to start the execution of
		\irace. If the number of initial configurations supplied in the file is less
		than the number of configurations required by \irace in the first iteration,
		additional configurations will be sampled uniformly at random.
		
		The format of the configurations file is one configuration per line, and one
		parameter value per column. The first line must give the parameter name
		corresponding to each column (names must match those given in the parameters
		file).  Each configuration must satisfy the parameter conditions (\code{NA}
		should be used for those parameters that are not enabled for a given
		configuration) and not be forbidden by the constraints that define forbidden
		configurations (\autoref{sec:forbidden}), if any.
		
		Figure~\ref{fig:acotsp_default} gives an example file that corresponds to the
		\ACOTSP scenario.
		
		\begin{figure}[!ht]
			\centering
			\begin{minipage}{0.7\linewidth}
				\footnotesize
				\begin{CodeInput}
					## Initial candidate configuration for irace
					algorithm localsearch alpha beta rho  ants nnls dlb q0 rasrank elitistants
					as        0           1.0   1.0  0.95 10   NA   NA  0  NA      NA
				\end{CodeInput}
			\end{minipage}
			\caption{Initial configuration file (\code{default.txt}) for tuning \ACOTSP.}\label{fig:acotsp_default}
		\end{figure}
		
		We advise to use this feature when a default configuration of the target algorithm exists or when
		different sets of good parameter values are known. This will allow \irace to start the search
		from those parameter values and attempt to improve their performance.
		
		
		\subsection{Forbidden configurations}\label{sec:forbidden}
		
		The scenario option \parameter{forbiddenFile} specifies a text file containing
		logical expressions of parameter values that valid configurations should not
		satisfy, that is, no configuration that satisfies any of these logical
		expressions will be evaluated by \irace. This is useful when some combination
		of parameter values could cause the target algorithm to crash, consume
		excessive CPU time or memory, or when it is known that they do no produce
		satisfactory results.
		
		The format of the forbidden configurations file is one constraint per line,
		where each constraint is a logical expression (in \aR syntax) containing
		parameter names as defined by the \parameter{parameterFile}
		(\autoref{sec:target parameters}), values and logical operators. For a list of \aR logical operators see:
		\begin{center}
			\url{https://stat.ethz.ch/R-manual/R-devel/library/base/html/Syntax.html}
		\end{center}
		
		If a parameter configuration is generated that makes any of the logical
		expressions evaluate to \code{TRUE}, then the configuration is considered
		forbidden and it is discarded. Figure~\ref{fig:acotsp_forbidden} shows an
		example file that corresponds to the \ACOTSP scenario.
		
		\begin{figure}[!ht]
			\centering
			\begin{minipage}{0.5\linewidth}
				\begin{CodeInput}
					## Examples of valid logical operators are:
					## ==  !=  >=  <=  >  <  &  |  !  %in%
					(alpha == 0.0) & (beta == 0.0)
				\end{CodeInput}
			\end{minipage}
			\caption{Forbidden configurations file (\code{forbidden.txt}) for tuning \ACOTSP.}\label{fig:acotsp_forbidden}
		\end{figure}
		
		\begin{xwarningbox}
			If initial configuration are provided (\autoref{sec:initial}), they must also comply with the
			constraints defined in \parameter{forbiddenFile}.
		\end{xwarningbox}
		
		\begin{xwarningbox}
			Categorical and ordinal parameters are always treated as strings. Given a parameter like:
			\begin{CodeInput}
				a "" c (0, 5, 10, 20)
			\end{CodeInput}
			then, a condition like \code{a > 10} will be true when \code{a} is \code{5},
			because comparisons between strings are lexicographic and \code{"10"} is sorted
			before \code{"5"}. As a work-around, you can convert the string to numeric in
			the condition with \code{as.numeric(a)}.
		\end{xwarningbox}
		
		\subsection{Repairing configurations}\label{sec:repairconf}
		
		In some problems, the parameter values require complex constraints that cannot
		be implemented by constraints defined in \parameter{forbiddenFile}
		(\autoref{sec:forbidden}).
		%
		The scenario option \parameter{repairConfiguration} can be set to a
		user-defined \aR function that takes a single configuration generated by irace
		and returns a ``\emph{repaired}'' configuration, thus allowing the
		implementation of any rules necessary to satisfy arbitrary constraints on
		parameter values.
		%
		The \parameter{repairConfiguration} function is called after generating a
		configuration and before checking for forbidden configurations. The first
		argument is a 1-row \code{data.frame} with parameter names as the column names,
		the second argument is the \code{parameters} list (\autoref{sec:parameters
			object}), and the third argument is the scenario
		variable \parameter{digits}. An example that makes all real-valued parameters
		sum up to one would be:
		%
		<<repairEx,prompt=FALSE, eval=FALSE>>=
		repairConfiguration = function (configuration, parameters, digits)
		{
			isreal <- parameters$type[colnames(configuration)] %in% "r"
			configuration[isreal] <- configuration[isreal] / sum(configuration[isreal])
			return(configuration)
		}
		@
		The following example forces  three specific parameters to be in increasing order:
		%
		<<repairEx2,prompt=FALSE, eval=FALSE>>=
		repairConfiguration = function (configuration, parameters, digits)
		{
			columns <- c("p1","p2","p3")
			# cat("Before"); print(configuration)
			configuration[columns] <- sort(configuration[columns])
			# cat("After"); print(configuration)
			return(configuration)
		}
		@
		
		The above code can be specified directly in the \parameter{scenarioFile}, by default \code{scenario.txt}.
		
		%%
		%%
		%%
		%% Parallel processing
		%%
		%%
		%%
		
		\section{Parallelization}\label{sec:parallel}
		A single run of \irace can be done much faster by executing the calls
		to \parameter{targetRunner} (the runs of the target algorithm) in
		parallel. There are four ways to parallelize a single run of \irace:
		
		\begin{enumerate}
			\item \textbf{Parallel processes}: The option \parameter{parallel} allows
			executing in parallel, within a single computer, the calls
			to \parameter{targetRunner}, by means of the \pkg{parallel} \aR package. For
			example, adding \code{--parallel N} to the command line of \irace will launch
			in parallel up to $N$ calls of the target algorithm.
			
			\item \textbf{MPI}: By enabling the option \parameter{mpi}, calls
			to \parameter{targetRunner} will be executed in parallel by using the message
			passing interface (MPI) protocol (requires the \pkg{Rmpi} \aR package). In
			this case, the option \parameter{parallel} controls the number of slave nodes
			used by \irace. For example, adding \code{--mpi 1 --parallel N} to the
			command-line will create $N$ slaves + 1 master, and execute up to $N$ calls of
			\parameter{targetRunner} in parallel.
			
			The user is responsible for setting up the required MPI environment.  MPI is
			commonly available in computing clusters and requires launching \irace in
			some particular way. An example script for using MPI mode in a SGE cluster
			is given at \IRACEHOME{/bin/parallel-irace-mpi}.
			
			By default, \irace dynamically balances the load among nodes, however, this
			may significantly increase communication overhead in some parallel
			environments, where disabling \parameter{loadBalancing} may be faster.
			
			\item \textbf{Batch jobs clusters}: Some computing clusters work by submitting
			jobs to a batch queue and waiting for the jobs to finish. With the
			option \parameter{batchmode} (\code{--batchmode} \code{[sge|pbs|} \code{torque|slurm]}),
			\irace will launch in parallel as many calls of \parameter{targetRunner} as
			possible (\parameter{parallel} can be used to set a limit) and use a
			cluster-specific method to wait for jobs to finish. If your cluster type is
			not supported or not working as expected, please contact us and we will
			gladly add support for it.
			
			\begin{xwarningbox}
				In this mode, \irace must run in the submission node of the cluster, and
				hence, \irace should not be submitted to the cluster as a job (that is,
				neither \code{qsub} nor \code{squeue} should be used to invoke \irace
				itself). The user must call the appropriate job submission command (e.g., \code{qsub}) from within
				\parameter{targetRunner} with the appropriate settings for their cluster,
				otherwise \parameter{targetRunner} will not submit jobs to the
				cluster. The script must return a single string: The job ID that allows
				\irace to determine the status of the running job. Moreover, the use of a
				separate
				\parameter{targetEvaluator} script is required to evaluate the results of
				\parameter{targetRunner} and return them to \irace.
			\end{xwarningbox}
			See the examples in
			\IRACEHOME{/examples/batchmode-cluster/}.
			
			\item \parameter{targetRunnerParallel}: This option allows users to fully
			control the parallelization of the execution of \parameter{targetRunner}. 
			Its value must be an \aR function that will be invoked by \irace as follows:
			
			<<targetRunnerParallel,prompt=FALSE, eval=FALSE>>=
			targetRunnerParallel(experiments, exec.target.runner, scenario, target.runner)
			@
			%
			where \code{scenario} is the list describing the configuration scenario
			(\autoref{sec:scenario}); \code{experiments} is a list that describes the
			configurations and instances to be executed (see \autoref{sec:runner} for a
			description); \code{target.runner} is the function that calls the target
			algorithm and it is the same as \parameter{targetRunner}, if the latter is a
			function, or it is a call to \code{target.runner.default}, if
			\parameter{targetRunner} is the path to an executable; and
			\code{exec.target.runner} is an internal function within \irace that takes
			care of executing \code{target.runner}, check its output and, possibly,
			retry in case of error (see \parameter{targetRunnerRetries}). The
			\parameter{targetRunnerParallel} function should call the given
			\code{target.runner} function for each element in the \code{experiments}
			list, possibly using \code{exec.target.runner} as a wrapper. A trivial
			example would be:
			
			<<targetRunnerParallel2,prompt=FALSE, eval=FALSE>>=
			targetRunnerParallel <- function(experiments, exec.target.runner, scenario)
			{
				return (lapply(experiments, exec.target.runner, scenario = scenario,
				target.runner = target.runner))
			}
			@
			%
			However, the user is free to set up the calls in any way, perhaps
			implementing its own replacement for \code{target.runner} and/or
			\code{exec.target.runner}.
			
			The only requirement is that the \parameter{targetRunnerParallel} function
			must return a list of the same length as \code{experiments}, where each
			element is the output expected from the corresponding call
			to \parameter{targetRunner} (see \autoref{sec:runner}). The following is an
			example of the output of a call to \parameter{targetRunnerParallel} with 2
			experiments, in which the execution time is not reported:
			
			<<targetRunnerParallel3,prompt=FALSE, eval=TRUE>>=
			print(output)
			@
			%
			
		\end{enumerate}
		
		%%
		%%
		%%
		%% Testing
		%%
		%%
		%%
		
		\section{Testing (Validation) of configurations}\label{sec:testing}
		
		Once the tuning process is finished, \irace returns a set of configurations
		corresponding to the elite configurations at the end of the run, ordered from
		best to worst. In order to evaluate the generality of these configurations
		without looking at their performance on the training set, \irace offers the
		possibility of evaluating these configurations on a test instance set,
		typically different from the training set used during the tuning phase. These
		evaluations will use the same settings for parallel
		execution, \parameter{targetRunner} and \parameter{targetEvaluator}.
		
		The test instance set can be specified by the
		options \parameter{testInstancesDir} and \parameter{testInstancesFile}, or by
		setting directly the variable \code{scenario\$testInstances}, which behave the
		same as their counterparts for the training instances
		(\autoref{sec:training}). In particular, each test instance is assigned a
		different seed in the same way as done for the training instances. In
		principle, \irace evaluates each configuration on each testing instance just
		once, because evaluating one run on $n$ instances is always better than
		evaluating $n'$ runs on $n/n'$ instances~\citep{IRIDIA-2004-001}. However, if
		the number of instances is limited, one can always duplicate instances as
		needed in the \parameter{testInstancesFile}, and \irace will assign a different
		random seed to each instance.
		
		
		The options \parameter{testNbElites} and \parameter{testIterationElites}
		control which configurations are evaluated during the testing phase. In
		particular, setting \code{testIterationElites = 1} will test not only the final
		set of elite configurations (those returned at the end of the training phase),
		but also the set of elites at the end of each race
		(iteration). The option \parameter{testNbElites} limits the maximum number of 
		configurations considered within each set. Some examples:
		\begin{itemize}
			\item \code{testIterationElites = 0; testNbElites = 1} means that only the best
			configuration found during the run of \irace, the final best, will be used in
			the testing phase.
			\item \code{testIterationElites = 1; testNbElites = 1} will test, in addition to the final best, the best configuration found at each iteration.
			\item \code{testIterationElites = 1; testNbElites = 2} will test the two best
			configurations found at each iteration, in addition to the final best and
			second-best configurations.
		\end{itemize}
		
		The testing can be also (re-)executed at a later time by using the following
		\aR command:
		%
		<<testing_r, prompt=FALSE, eval=FALSE>>=
		testing.main(logFile = "./irace.Rdata")
		@
		%
		The above line will load the scenario setup from \parameter{logFile} to
		perform the testing.  The testing results will be stored in the \aR object
		\code{iraceResults\$testing}, which is saved in the file specified by
		\code{scenario\$logFile}. The structure of the object is described in
		\autoref{sec:output r}.  For examples on how to analyse the results see
		\autoref{sec:analysis}.
		
		Another alternative is to test a specific set of configurations using the command-line option \parameter{-{}-only-test} as follows:
		\begin{lstlisting}[style=BashInputStyle]
			irace --only-test configurations.txt
		\end{lstlisting}
		where \path{configurations.txt} has the same format as the 
		set of initial configurations (\autoref{sec:initial}).
		
		%%
		%%
		%%
		%% Recovery
		%%
		%%
		%%
		\section{Recovering \irace runs}\label{sec:recovery}
		
		Problems like power cuts, hardware malfunction or the need to use computational
		power for other tasks may occur during the execution of \irace, terminating a
		run before completion. At the end of each iteration, \irace saves an \aR data
		file (\parameter{logFile}, by default \code{"./irace.Rdata"}) that not only
		contains information about the tuning progress (\autoref{sec:output r}), but
		also internal information that allows recovering an incomplete execution.
		
		To recover an incomplete \irace run, set the option \parameter{recoveryFile} to
		the log file previously produced, and \irace will continue the execution from
		the last saved iteration. The state of the random generator is saved and
		loaded, therefore, as long as the execution is continued in the same machine,
		the obtained results will be exactly the same as executing \irace in one step
		(external factors, such as CPU load and disk caches, may affect the
		target algorithm and that may affect the results).  You can specify
		the \parameter{recoveryFile} from the command-line or from the scenario file,
		and execute \irace as described in \autoref{sec:execution}. For example, from
		the command-line use:
		
		\begin{lstlisting}[style=BashInputStyle]
			irace --recovery-file "./irace-backup.Rdata"
		\end{lstlisting}
		
		
		\begin{xwarningbox}
			When recovering a previous run, \irace will try to save data on the file
			specified by the \parameter{logFile} option. Thus, you must specify different
			files for \parameter{logFile} and \parameter{recoveryFile}.  Before
			recovering, we strongly advise to rename the saved \aR data file as in the
			example above, which uses \code{"irace-backup.Rdata"}.
		\end{xwarningbox}
		
		\hide{%
			
			%\LESLIE{Actually, there other variables we allow to change...what we shoudl say about this?}
			% MANUEL: I would say nothing for now until we fix this properly.
			When recovery is done you can modify some \irace options from the command-line
			or from the scenario file.
			
			\begin{itemize}
				\item \parameter{execDir}
				\item \parameter{logFile}
				\item \parameter{debugLevel}
				\item \parameter{parallel}
				\item \parameter{loadBalancing}
				\item \parameter{mpi}
				\item \parameter{batchmode}
			\end{itemize}
			
			It is not possible (and in most cases not correct) to change the scenario options, given that the
			previous results can become invalid. However there are some cases in which this options must be changed,
			for example, if you have instances that are files, it might be the case that the instances are not
			available at the same location. To change the location of them you should modify directly the \aR data file,
			please be careful not to change the order of the instances, because this would make the results obtained by
			\irace invalid. To change the instances path, open the \aR console and use:
			
			<<change_recover, prompt=TRUE, eval=FALSE>>=
			load ("~/tuning/irace.Rdata")
			new.path <- "~/experiments/tuning/instances/"
			iraceResults$scenario$instances <-
			paste (new.path,
			basename(iraceResults$scenario$instances),
			sep="")
			save (iraceResults, file="~/tuning/irace.Rdata")
			@
			
			This example can also can be applied to the target execution files \parameter{targetRunner} and \parameter{targetEvaluator}.
		}
		
		
		\begin{xwarningbox}
			Do not change anything in the log file or the scenario file before
			recovering, as it may have unexpected effects on the recovered run of
			\irace. In case of doubt, please contact us first (\autoref{sec:contact}). In particular, it is not possible to continue a run of \irace by recovering with a larger budget. Results will \textbf{not} be the same as running \irace from the start with the largest budget.  An alternative is to use the final configurations from one run as the initial configurations of a new run.
		\end{xwarningbox}
		
		\begin{xwarningbox}
			If your scenario uses \parameter{targetEvaluator} (\autoref{sec:evaluator})
			and \parameter{targetEvaluator} requires files created
			by \parameter{targetRunner}, then recovery will fail if those files are not
			present in the \parameter{execDir} directory. This can happen, for example,
			if you recover from a different directory than the one from which irace was
			initially executed, or when \parameter{execDir} is set to a temporary
			directory for every irace run. Thus, you need to copy the contents of the
			previous \parameter{execDir} into the new one.
		\end{xwarningbox}
		
		%%
		%%
		%%
		%% Output and analysis of results
		%%
		%%
		%%
		\section{Output and results}
		
		During its execution, \irace prints information about the progress of the
		tuning in the standard output.  Additionally, after each iteration, an \aR data
		file is saved (\parameter{logFile} option) containing the state of \irace.
		
		\subsection{Text output}\label{sec:output text}
		
		Figure~\ref{fig:output} shows the output, up to the end of the first iteration,
		of a run of elitist \irace applied to the \ACOTSP scenario with 1000
		evaluations as budget.
		
		\afterpage{\clearpage
			\begin{figure}[p]
				\RecustomVerbatimCommand{\VerbatimInput}{VerbatimInput}%
				{fontsize=\scriptsize,
					%
					frame=none,  % top and bottom rule only
					framesep=0pt, % separation between frame and text
					rulecolor=\color{Gray},
					firstline=1,
					lastline=68,
				}
				\VerbatimInput{irace-acotsp-stdout.txt}%
				\caption{Sample text output of \irace.}\label{fig:output}
		\end{figure}}
		
		First, \irace gives the user a warning informing that it has found a file with
		the default scenario filename and it will use it. Then, general information
		about the selected \irace options is printed:
		\begin{itemize}
			\item \parameter{nbIterations} indicates the minimum number of iterations \irace has calculated
			for the scenario. Depending on the development of the tuning the final iterations that
			are executed can be more.
			\item \parameter{minNbSurvival} indicates the minimum number of alive configurations that
			are required to continue a race. When less configurations are alive the race is
			stopped and a new iteration begins.
			\item \code{nbParameters} is the number of parameters of the scenario.
			\item \parameter{seed} is the number that was used to initialize the random number generator in \irace.
			\item \code{confidence level} is the confidence level of the statistical test.
			\item \code{budget} is the total number of evaluations available for the tuning.
			\item \code{time budget} is the  maximum execution time available for the tuning.
			\item \parameter{mu} is a value used for calculating the minimum number of iterations.
			\item \parameter{deterministic} indicates if the target algorithm is assumed to be deterministic.
		\end{itemize}
		
		
		At each iteration, information about the progress of the execution is printed as follows:
		\begin{itemize}
			\item \code{experimentsUsedSoFar} is the number of experiments from the total budget
			that have been used up to the current iteration.
			\item \code{timeUsed} is the execution time used so far in the experiments. Only available when reported
			in the \parameter{targetRunner} (activate it with the \parameter{maxTime} option).
			\item \code{remainingBudget} is the number of experiments that have not been used yet.
			\item \code{timeEstimate} estimation of the mean execution time. This is used to calculate 
			the remaining budget when \parameter{maxTime} is used.
			\item \code{currentBudget} is the number of evaluations \irace has allocated to the
			current iteration.
			\item \parameter{nbConfigurations} is the number of configurations \irace will use in the
			current iteration. In the first iteration, this number of configurations include
			the initial configurations provided; in later iterations, it includes the elite
			configurations from the previous iterations.
		\end{itemize}
		
		After the iteration information, a table shows the progress of the iteration execution.
		Each row of the table gives information about the execution of an instance in the race.
		The first column contains a symbol that describes the results of the
		statistical test:
		%
		\begin{description}
			\item \code{|x|} No statistical test was performed for this instance. The options \parameter{firstTest}
			and \parameter{eachTest} control on which instances the statistical test is performed.
			\item \code{|-|} Statistical test performed and configurations have been discarded. The column \code{Alive} gives an indication of how
			many configurations have been discarded.
			\item \code{|=|} Statistical test performed and no configurations have been discarded.
			This means \irace needs to evaluate more instances to identify the best configurations.
			\item \code{|!|} This indicator exists only for the elitist version of \irace. It indicates that
			the statistical test was performed and some elite configurations appear to show bad performance and
			could be discarded but they are kept because of the elitist rules. See option \parameter{elitist} in \autoref{sec:irace options} for more information.
		\end{description}
		
		Other columns have the following meaning:
		\begin{description}
			\item \code{Instance}: Index of \code{(instance, seed)} pair executed. This number
			corresponds to the index of the list found in \code{state$.irace$instancesList}. See \autoref{sec:output r}
			for more information. This is different from the instance ID passed to \parameter{targetRunner}.
			\item \code{Bound}: Only when \parameter{capping} is enabled. Execution time used as bound
			for the execution of new candidate configurations. 
			\item \code{Alive}: Number of configurations that have not been
			discarded after the statistical test was performed.
			\item \code{Best}: ID of the best configuration according to the instances seen
			so far in this race (i.e., not including previous iterations).
			\item \code{Mean best}: Mean of the best configuration across the instances seen so
			far in this race.
			\item  \code{Exp so far}: Number of experiments
			performed so far.
			\item \code{W time}:  Wall-clock time spent on this instance.
			\item \code{rho}, \code{KenW}, and \code{Qvar}: Spearman's rank correlation
			coefficient rho, Kendall's concordance coefficient W, and a variance measure
			described in \cite{SchHoo2012quanti}, respectively, of the configurations
			across the instances evaluated so far in this iteration.  These measures
			evaluate how consistent is the performance of the configurations across the
			instances.  Values close to 1 for \code{rho} and \code{KenW} and values close
			to 0 for \code{Qvar} indicate that the scenario is highly homogeneous. For
			heterogeneous scenarios, we provide advice in \autoref{sec:het}.
		\end{description}
		
		Finally, \irace outputs the best configuration found and a list of the elite
		configurations. The elite configurations are configurations that did not show
		statistically significant difference during the race; they are ordered
		according to their mean performance on the executed instances.
		
		
		\subsection[R data file (logFile)]{\aR data file (\code{logFile})}\label{sec:output r}
		
		The \aR data file created by \irace (by default as \code{irace.Rdata}, see option \parameter{logFile}) contains an object called
		\code{iraceResults}. You can load this file in the \aR console with:
		
		<<load_rdata, prompt=FALSE, eval=FALSE>>=
		load("irace-acotsp.Rdata")
		@
		
		The \code{iraceResults} object is a list, and the elements of a list can be
		accessed in \aR by using the \code{\$} or \code{[[]]} operators:
		
		<<show_version, prompt=TRUE, eval=TRUE, comment="">>=
		iraceResults$irace.version
		iraceResults[["irace.version"]]
		@
		
		\noindent%
		The \code{iraceResults} list contains the following elements:
		
		\begin{itemize}
			\item \code{scenario}: The scenario \aR object containing the \irace options
			used for the execution. See \autoref{sec:irace options} and the help
			of the \pkg{irace} package; open an \aR console and type: \code{?defaultScenario}.
			See \autoref{sec:irace options} for more information.
			
			\item \code{parameters}: The parameters \aR object containing the description of
			the target algorithm parameters. See \autoref{sec:target parameters}.
			
			\item \code{allConfigurations}: The target algorithm configurations generated by
			\irace. This object is a \code{data frame}, each row is a candidate
			configuration; the first column (\code{.ID.}) indicates the internal identifier of the
			configuration; the final column (\code{.PARENT.}) is the identifier
			of the configuration from which the current configuration was sampled; and the remaining columns correspond to the parameter values; each column is named as
			the parameter name specified in the parameter object. 
			
			<<show_configurations, prompt=TRUE, eval=TRUE, comment="">>=
			head(iraceResults$allConfigurations)
			@
			
			\item \code{allElites}: A list that contains one element per iteration. Each element contains
			the internal identifier of the elite candidate configurations of the corresponding iteration
			(identifiers correspond to \code{allConfigurations$.ID.}).
			
			<<show_idelites, prompt=TRUE, eval=TRUE, comment="">>=
			print(iraceResults$allElites)
			@
			
			The configurations are ordered by mean performance, that is, the ID of the best
			configuration corresponds to the first ID. To obtain the values of the
			parameters of all elite configurations found by \irace use:
			
			<<get_elites, prompt=TRUE, eval=TRUE, comment="">>=
			getFinalElites(logFile = "irace-acotsp.Rdata", n = 0)
			@
			
			\item \code{iterationElites}: A vector containing the best candidate configuration ID of each iteration. The best configuration found corresponds to the last one
			of this vector.
			<<show_iditelites, prompt=TRUE, eval=TRUE, comment="">>=
			print(iraceResults$iterationElites)
			@
			
			One can obtain the full configuration with:
			
			<<get_elite, prompt=TRUE, eval=TRUE, comment="">>=
			last <- length(iraceResults$iterationElites)
			id <- iraceResults$iterationElites[last]
			getConfigurationById(logFile = "irace-acotsp.Rdata", ids = id)
			@
			
			\item \code{rejectedConfigurations}: A vector containing the rejected configurations IDs. These 
			correspond to configurations that produced failed executions and were ignored by \irace during the
			configuration process. See \autoref{sec:reject} to enable the detection of such configurations.
			
			\item \code{experiments}: A matrix with configurations as columns and instances
			as rows. Column names correspond to the internal identifier of the configuration
			(\code{allConfigurations$.ID.}). The results of a particular
			configuration can be obtained using:
			
			<<get_experiments, prompt=TRUE, eval=TRUE, comment="">>=
			# As an example, we use the best configuration found
			best.config <- getFinalElites(iraceResults = iraceResults, n = 1)
			id <- best.config$.ID.
			# Obtain the configurations using the identifier
			# of the best configuration
			all.exp <- iraceResults$experiments[,as.character(id)]
			all.exp[!is.na(all.exp)]
			@
			
			When a configuration was not executed on an instance, its value is \code{NA}. A
			configuration may not be executed on an instance because: 1) it was not created
			yet when the instance was used, or 2) it was
			discarded by the statistical test and not executed on subsequent instances, or
			3) the race terminated before this instance was considered.
			
			Row names correspond to the identifier of the \code{(instance, seed)} pairs
			defined in \code{state\$.irace\$instancesList}. The instance and seed used
			for a particular experiment can be obtained with:\sloppy
			
			<<get_instance_seed, prompt=TRUE, eval=TRUE, comment="">>=
			# As an example, we get seed and instance of the experiments
			# of the best candidate.
			# Get index of the instances
			pair.id <- names(all.exp[!is.na(all.exp)])
			index <- iraceResults$state$.irace$instancesList[pair.id,"instance"]
			# Obtain the instance names
			iraceResults$scenario$instances[index]
			# Get the seeds
			iraceResults$state$.irace$instancesList[index,"seed"]
			@
			\item \code{experimentLog}: A matrix with columns \code{iteration, instance, configuration}. This matrix contains the log of all the experiments that \irace
			performs during its execution. The \code{instance} column refers to the index of the
			\code{state$.irace$instancesList} data frame. When \parameter{capping} is enabled a column \code{bound} is added to log the execution bound applied 
			for each execution.
			
			\item \code{softRestart}: A logical vector that indicates if a soft restart was
			performed on each iteration. If \code{FALSE}, then no soft restart was performed.
			See option \parameter{softRestart} in \autoref{sec:irace options}.
			
			\item \code{state}: A list that contains the state of \irace, the recovery
			(\autoref{sec:recovery}) is done using the information contained in this object. The
			probabilistic model of the last elite configurations can be found here by doing:
			
			<<get_model, prompt=TRUE, eval=TRUE, comment="">>=
			# As an example, we get the model probabilities for the
			# localsearch parameter.
			iraceResults$state$model["localsearch"]
			# The order of the probabilities corresponds to:
			iraceResults$parameters$domain$localsearch
			@
			
			The example shows a list that has one element per elite configuration (ID as
			element name). In this case, \code{localsearch} is a categorical parameter and 
			it has a probability for each of its values.
			
			
			\item \code{testing}: A list that contains the testing results.
			The list contains the following elements:
			\begin{itemize}
				\item \code{experiments}: Matrix of experiments in the same format as the
				\code{iraceResults$experiments} matrix. The column names indicate the candidate
				configuration identifier and the row names contain the name of the instances.
				<<get_test_exp, prompt=TRUE, eval=TRUE, comment="">>=
				# Get the results of the testing
				iraceResults$testing$experiments
				@
				
				\item \code{seeds}: The seeds used for the experiments, each seed corresponds to
				each instance in the rows of the test \code{experiments} matrix.
				<<get_test_seeds, prompt=TRUE, eval=TRUE, comment="">>=
				# Get the seeds used for testing
				iraceResults$testing$seeds
				@
				
				In the example, instance \code{1000-1.tsp} is executed with seed \Sexpr{iraceResults$testing$seeds[1]}.
				
			\end{itemize}
			
		\end{itemize}
		
		\subsection{Analysis of results}\label{sec:analysis}
		
		The final configurations returned by \irace are the elites of the final race.
		They are reported in decreasing order of performance, that is, the best
		configuration is reported first.
		
		If testing is performed, you can further analyze the resulting best
		configurations by performing statistical tests in \aR or just plotting the
		results:
		
		<<plot_test, fig.pos="tbp", fig.align="center", fig.height = 4, fig.width = 8, out.width='0.85\\textwidth', fig.cap="Boxplot of the testing results of the best configurations.", prompt=TRUE, eval=TRUE, comment="">>=
		results <- iraceResults$testing$experiments
		# Wilcoxon paired test
		conf <- gl(ncol(results), # number of configurations
		nrow(results), # number of instances
		labels = colnames(results))
		pairwise.wilcox.test (as.vector(results), conf, paired = TRUE, p.adj = "bonf")
		# Plot the results
		configurationsBoxplot (results, ylab = "Solution cost")
		@
		
		%%FIXME: should we add something like this?
		%The Kendall concordance coefficient (\code{W}) and the Spearman's rho can be
		%applied over data that has the characteristics of the data obtained in the testing,
		%that is a full matrix where all configurations are executed in all instances. \code{W}
		%can show if the configurations tested have an homogeneous performance on the used instances
		%set. If evidence of an heterogeneous scenario found we recommend to make some adjustments
		%in the \irace options as described in \autoref{sec:}
		
		%<<conc, prompt=TRUE, eval=TRUE, comment="">>=
		%irace:::concordance(iraceResults$testing$experiments)
		%@
		%
		
		During the tuning, \irace iteratively updates the sampling models of the
		parameters to focus on the best regions of the parameter search space. The
		frequency of the sampled configurations can provide insights on the parameter
		search space. We provide a function for plotting the
		frequency of the sampling of a set of configurations. For more information on this function, please see the \aR help, type in the
		\aR console: \code{?parameterFrequency}. The following example
		plots the frequency of the parameters sampled during one \irace run:
		
		<<freq, fig.pos="tbp", fig.cap="Parameters sampling frequency.", out.width="\\textwidth",prompt=TRUE, eval=TRUE, comment="">>=
		parameterFrequency(iraceResults$allConfigurations, iraceResults$parameters)
		@
		
		By using parallel coordinates plots, it is possible to analyze how the parameters
		interact with each other. For more information on this function, please see the \aR help, type
		in the \aR console: (\code{?parallelCoordinatesPlot}).
		The following example shows how to create a parallel
		coordinate plot of the configurations in the last two iterations of \irace.
		
		<<parcord, fig.pos="tbp", fig.align="center", out.width="0.7\\textwidth", fig.cap="Parallel coordinate plots of the parameters of the configurations in the last two iterations of a run of \\irace.", prompt=FALSE, eval=TRUE, comment="">>=
		# Get last iteration number
		last <- length(iraceResults$iterationElites)
		# Get configurations in the last two iterations
		conf <- getConfigurationByIteration(iraceResults = iraceResults,
		iterations = c(last - 1, last))
		parallelCoordinatesPlot (conf, iraceResults$parameters,
		param_names = c("algorithm", "alpha", "beta", "rho", "q0"),
		hierarchy = FALSE)
		@
		
		%%% FIXME: This plot is completely misleading. We should either calculate the
		%%% performance over the whole training set, or find a way to plot the
		%%% performance estimated by irace as a confidence interval to account for
		%%% different number of instances. Or simply plot a heatmap of instances x
		%%% best-of-each iteration.
		\hide{It is also possible to plot the performance evolution of the best-so-far configuration over the number of experiments as follows:
			%% <<trainEvo, fig.pos="tbp", fig.align="center", out.width='0.75\\textwidth', fig.cap="Training set performance of the best-so-far configuration over number of experiments.", prompt=FALSE, eval=TRUE, comment="">>=
			# Get number of iterations
			iters <- unique(iraceResults$experimentLog[, "iteration"])
			# Get number of experiments (runs of target-runner) up to each iteration
			fes <- cumsum(table(iraceResults$experimentLog[,"iteration"]))
			# Get the mean value of all experiments executed up to each iteration
			# for the best configuration of that iteration.
			values <- sapply(iters, function(k) {
				instances <- as.character(
				unique(iraceResults$experimentLog[
				iraceResults$experimentLog[, "iteration"] == k,
				"instance"]))
				return(mean(iraceResults$experiments[
				instances,
				as.character(iraceResults$iterationElites[k])]))})
			plot(fes, values, type="s", xlab = "Number of runs of the target algorithm",
			ylab = "Estimated mean value over whole training set")
			points(fes, values)
		}
		
		It is also possible to plot the performance on the test set of the best-so-far configuration over the number of experiments as follows:
		<<testEvo, fig.pos="tbp", fig.align="center", out.width='0.75\\textwidth', fig.cap="Testing set performance of the best-so-far configuration over number of experiments. Label of each point is the configuration ID.", prompt=FALSE, eval=TRUE, comment="">>=
		# Get number of iterations
		iters <- unique(iraceResults$experimentLog[, "iteration"])
		# Get number of experiments (runs of target-runner) up to each iteration
		fes <- cumsum(table(iraceResults$experimentLog[,"iteration"]))
		# Get the mean value of all experiments executed up to each iteration
		# for the best configuration of that iteration.
		elites <- as.character(iraceResults$iterationElites)
		values <- colMeans(iraceResults$testing$experiments[, elites])
		plot(fes, values, type = "s",
		xlab = "Number of runs of the target algorithm",
		ylab = "Mean value over testing set")
		points(fes, values)
		text(fes, values, elites, pos = 1)
		@
		
		The \irace package also provides an implementation of the ablation
		method~\cite{FawHoos2015ablation}. See \autoref{sec:ablation}.
		
		%%
		%%
		%%
		%% Advanced options
		%%
		%%
		%%
		
		\section{Advanced topics}
		
		\subsection{Tuning budget}\label{sec:budget}
		
		Before setting the budget for a run of \irace, please consider the number of
		parameters that need to be tuned, available processing power and available
		time. The optimal budget depends on the difficulty of the tuning scenario, the
		size of the parameter space and the heterogeneity of the instances. Typical
		values range from $1\,000$ to $100\,000$ runs of the target algorithm, although
		smaller and larger values are also possible. Currently, \irace does not detect
		whether the given budget allows generating all possible configurations. In such
		a case, the use of \emph{iterated} racing is unnecessary: One can simply
		perform a single race of all configurations (see FAQ in
		\autoref{faq:allconfs}).
		
		\Irace provides two options for setting the total tuning budget (\parameter{maxExperiments} and 
		\parameter{maxTime}). The option \parameter{maxExperiments} 
		limits the number of executions of \parameter{targetRunner} performed by \irace. The option \parameter{maxTime} limits the 
		total time of the \parameter{targetRunner} executions. When this latter option is used, \parameter{targetRunner} 
		must return the evaluation cost together with the execution time (\code{"cost time"}). 
		
		\begin{xwarningbox}
			When the goal is to minimize the computation time of an algorithm, and you wish to use \parameter{maxTime} as the tuning budget, \parameter{targetRunner} must return the time also as the evaluation cost, that is, return the time two times as \code{"time time"}.
		\end{xwarningbox}
		
		\begin{xwarningbox}
			When using \parameter{targetEvaluator} and using \parameter{maxTime} as tuning budget, \parameter{targetRunner} just returns the time (\code{"time"}) and \parameter{targetEvaluator} returns the cost.
		\end{xwarningbox}
		
		When using \parameter{maxTime}, \irace estimates the execution time of each \parameter{targetRunner} execution 
		before the configuration. The amount of budget used for the estimation is set with the option \parameter{budgetEstimation}
		(default is $2\%$). The obtained estimation is adjusted after each iteration using the obtained results and it is used to estimate the number of experiments that can be executed. Internally, \irace uses the number of remaining experiments to 
		adjust the number of configurations tested in each race. 
		
		\subsection{Multi-objective tuning}\label{sec:multi objective}
		
		Currently, \irace only optimizes one cost value at a time, which can be
		solution cost, computation time or any other objective that is returned to
		\irace by the \parameter{targetRunner}. If the target algorithm is
		multi-objective, it will typically return not a single cost value, but a set of
		objective vectors (typically, a Pareto front). For tuning such a target
		algorithm with \irace, there are two alternatives. If the algorithm returns a
		single vector of objective values, they can be aggregated into one single number
		by using, for example, a weighted sum. Otherwise, if the target algorithm
		returns a set of objective vectors, a unary quality metric (\eg~the
		hypervolume) may be used to evaluate the quality of the set.\footnote{An implementation is publicly available at \url{http://lopez-ibanez.eu/hypervolume}~\cite{FonPaqLop06:hypervolume}}
		
		% The first option is simple, it requires to devise a formula that can aggregate
		% the objectives in a way that balances the importance of all of them. This might
		% not be an easy task in some scenarios, and therefore using a more adequate
		% indicator to evaluate the performance of a multi-objective optimizer, such as
		% the hypervolume, is strongly advised.  
		
		%For using the hypervolume as evaluation \irace needs to postpone the evaluation of the
		%configurations in an instance until all the executions have been completed. Once
		%the executions are finalized, the evaluation starts by obtaining reference points for 
		%each objective from the configuration results. These points are used to define the 
		%Pareto front and the hypervolume of each configuration
		%is calculated and the tuning process continues normally.
		
		The use of aggregation or quality metrics often requires normalizing the
		different objectives. If normalization bounds are known a priori for each
		instance, normalized values can be computed by \parameter{targetRunner}.
		Otherwise, the bounds may be dynamically computed while running \irace, by
		using \parameter{targetEvaluator}. In this case, \parameter{targetRunner} will
		save the output of the algorithm, then the first call
		to \parameter{targetEvaluator} will examine the output produced by all calls
		to \parameter{targetRunner} for the same instance, update the normalization
		bounds and return the normalized output. Subsequent calls
		to \parameter{targetEvaluator} for the same instance will simply return the
		normalized output.
		%
		A similar approach can be used to dynamically compute the reference points or
		sets often required by unary quality metrics.
		
		For more information about defining a \parameter{targetEvaluator}, see
		\autoref{sec:evaluator}. Examples of tuning a multi-objective target algorithm
		using the hypervolume can be found in the examples at
		\IRACEHOME{/examples/hypervolume} and \IRACEHOME{/examples/moaco}.
		
		\subsection{Tuning for minimizing computation time}
		\label{sec:capping}
		
		When using \irace for tuning algorithms that report computation time to
		reach a target, \parameter{targetRunner} should return the execution time of a configuration instead of solution cost.
		
		%Even though \irace can be used for
		%minimizing computation time, \irace may itself require more time to do so in
		%its current version than other methods, such as
		%\code{ParamILS}\footnote{\url{http://www.cs.ubc.ca/labs/beta/Projects/ParamILS/}}
		%or \code{SMAC}\footnote{\url{http://www.cs.ubc.ca/labs/beta/Projects/SMAC/}},
		%since it does not make use of techniques, such as ``adaptive capping'', that
		%avoid long runs of the target algorithm.
		%We are currently extending \irace with an adaptive capping mechanism.
		
		Starting from version $3.0$,  \irace includes an elitist racing procedure that 
		implements an \textbf{adaptive capping mechanism} \citep{PerLopHooStu2017:lion}. 
		Adaptive capping \citep{HutHooLeyStu2009jair} is a configuration technique 
		that avoids the execution of long runs of the target algorithm, focusing the 
		configuration budget in the evaluation of the best configurations found. This is 
		done by bounding the execution time of each configuration based on the best 
		performing candidate configurations. 
		
		To use adaptive capping, the \parameter{capping} option must be enabled and the 
		\parameter{elitist} irace option must be selected. When evaluating candidate 
		configurations on an instance, \irace calculates an execution bound based on the 
		execution times of the elite configurations.  The \parameter{boundType} option defines 
		how the performance of the elite configurations  is defined to obtain the execution bound. 
		The default value of \parameter{boundType} calculates the performance ($p_i^s$) of each 
		elite configuration ($s$) as the mean execution time of the  instances already executed 
		in the race and the currently executed instance ($i$). The \parameter{cappingType} option 
		specifies the measure used to obtain the elite configurations bound. By default, the execution bound 
		is calculated as the median of the execution times of the elite configurations:
		%
		\begin{equation}
			b_i= \text{Median}_{\theta_s \in \Celite} \{ p_i^s\} 
		\end{equation}
		
		The execution bound for new configurations ($j$) is calculated by multiplying the elite configurations
		bound by the number of instances ($i$) in the execution list and subtracting the mean execution time 
		of the instances executed by the candidate:
		%
		\begin{equation}
			k_i^{'j} = b_{i} \cdot i + \bmin - p^{j}_{i-1} \cdot (i-1)
		\end{equation}
		
		A small constant \bmin is added to account for time measurements errors. 
		These settings are also used to apply a dominance elimination criterion together with the
		statistical test elimination. The domination criterion is defined as:
		%
		\begin{equation}\label{eq:dom}
			b_i + \bmin  < p_i^{j}
		\end{equation}
		
		When elite configurations dominate new configurations, these are eliminated from the race.
		
		\begin{xwarningbox}
			The default statistical test when \parameter{capping} is enabled is \code{t-test}. 
			This test is more appropriate to configure algorithms for optimizing runtime (see \autoref{sec:stat test}).
		\end{xwarningbox} 
		
		The execution bound is constantly adjusted by \irace based on the best configurations 
		times, nevertheless, a maximum execution time (\bmax) is never  exceeded. This maximum 
		execution time must be defined in the configuration scenario when \parameter{capping} 
		is enabled. To specify the maximum execution bound for the target runner executions 
		use the \parameter{boundMax} option. The final execution bound ($k_i^j$) is calculated by:
		%
		\begin{equation}\label{eq:cap}
			k_i^j = \begin{cases}
				\bmax &\text{if $k_i^{'j} > \bmax$,} \\
				\min \{b_i, \bmax\}& \text{if $k_i^{'j} \leq 0$,}\\
				k_i^{'j}&\text{otherwise;}
			\end{cases}
		\end{equation}
		
		Additionally, the \parameter{boundDigits} option defines the precision of the time bound
		provided by \irace, the default setting is 0.  
		
		Timed out executions occur when the maximum execution bound
		(\parameter{boundMax}) is reached and the algorithm has not achieved successful
		termination or a defined quality goal.  In this case, it is a common practice
		to apply a penalty known as PARX, in which timeouts are penalized by
		multiplying \parameter{boundMax} by a constant $X$. The constant $X$ may be set
		using the \parameter{boundPar} option. Bounded executions are executions that
		do not achieve successful termination or a defined quality goal in the
		execution bound ($k_i^j$) set by \irace, which is smaller than
		\parameter{boundMax}. The \parameter{boundAsTimeout} option replaces the
		evaluation of bounded executions by the \parameter{boundMax} value. More
		details about the implementation of adaptive capping can be found in
		\citet{PerLopHooStu2017:lion}.
		
		\begin{xwarningbox}
			Note that bounded executions are not timed out executions and thus, they will not be penalized by PARX. 
		\end{xwarningbox}
		
		
		\begin{xwarningbox}
			Penalized evaluations of timed out and bounded executions are only used for
			the elimination tests and the comparison between the quality of
			configurations. To calculate execution bounds and computation budget
			consumed, \irace uses only unpenalized execution times. The unpenalized
			execution time must be provided by the target runner or target evaluator as
			described in \autoref{sec:runner} and \autoref{sec:evaluator} .
		\end{xwarningbox}
		
		
		
		\subsection{Hyper-pararameter optimization of machine learning methods}
		
		The \irace package can also be used for model selection and hyper-parameter optimization 
		of machine learning methods. For such a task, we recommend the \pkg{mlr} package~\citep{R:mlr}. 
		The following webpage documents how to use \irace for this purpose: \url{https://mlr-org.github.io/mlr-tutorial/devel/html/advanced_tune/index.html}
		
		
		\subsection{Heterogeneous scenarios}
		\label{sec:het}
		
		We classify a scenario as homogeneous when the target algorithm has a
		consistent performance regarding the instances; roughly speaking, good
		configurations tend to perform well and bad configurations tend to perform
		poorly on all instances of the problem. By contrast, in heterogeneous
		scenarios, the target algorithm has an inconsistent performance on different
		instances, that is, some configurations perform well for a subset of the
		instances, while they perform poorly for a different subset.
		
		When facing a heterogeneous scenario, the first question should be whether the
		objective of tuning is to find configurations that perform reasonably well over
		all instances, even if they are not the best ones in any of them. If this is
		not the case, then it would be better to partition instances into more similar
		subsets and execute \irace separately on each subset. This will lead to a
		portfolio of algorithm configurations, one for each subset, and algorithm
		selection techniques can be used to select the best configuration from the
		portfolio when facing a new instance.
		
		If finding an overall good configuration for all the instances is the
		objective, then we recommend that instances are randomly sampled
		(option \parameter{sampleInstances}), unless one can provide the instances in a
		particular order that does not bias the tuning towards any subset.  For
		example, let's assume a heterogeneous scenario with two classes of instances.  If
		training instances are not sampled and the first ten instances belong to only
		one class, the tuning will be biased towards configurations that perform good
		for those instances.  An optimal order would not ever present consecutively two
		instances of the same type.
		
		In addition, it may be useful to increase the number of instances executed
		before doing a statistical test in order to see more instance classes before
		discarding configurations. The option \parameter{elitistNewInstances} in elitist
		\irace (option \parameter{elitist}) can be used to increase the number of new
		instances executed in each iteration, \eg \code{--elitist-new-instances 5} (default
		value is 1). For the non-elitist \irace, the option \parameter{firstTest} may be used for the same purpose, \eg \code{--first-test 10} (default value is 5).
		\MANUEL{And both at the same time?}
		
		While executing \irace, the homogeneity of the scenario can be observed by
		examining the values of Spearman's rank correlation coefficient and Kendall's
		concordance coefficient in the text output of \irace.  See \autoref{sec:output
			text} for more information.
		
		\subsection{Choosing the statistical test}
		\label{sec:stat test}
		
		The statistical test used in \irace identifies statistically bad performing
		configurations that can be discarded from the race in order to save
		budget. Different statistical tests use different criteria to compare the
		cost of the configurations, which has an effect on the tuning results.
		
		\Irace provides two types of statistical tests
		(option \parameter{testType}). Each test has different characteristics that are
		beneficial for different goals:
		%
		\begin{itemize}
			\item Friedman test (\code{F-test}): This test uses the ranking of the
			configurations to analyze the differences between their performance. This
			makes the test suitable for scenarios where the scale of the performance
			metric is not as important to assess configurations as their relative
			ranking.  This test is also indicated when the distribution of the mean
			performances deviates greatly from a normal distribution. For example, the
			ranges of the performance metric on different instances may be completely
			difference and comparing the performance of different configurations using
			the mean over multiple instances may be deceiving.  We recommend to use the
			\code{F-test} (default when \parameter{capping} is not enabled) when tuning
			for solution cost and whenever the best performing algorithm should be among
			the best in as many instances as possible.
			
			\item Student's t-test (\code{t-test}): This test uses the mean performance of
			the configurations to analyze the differences between the
			configurations.\footnote{The t-test does not require that the performance
				values follow a normal distribution, only that the distribution of sample
				means does. In practice, the t-test is robust despite large deviations from
				the assumptions.} This makes the test suitable for scenarios where the
			differences between values obtained for different instances are relevant to
			assess good configurations.  We recommend using t-test, in particular, when
			the target algorithm is minimizing computation time and, in general, whenever
			the best configurations should obtain the best average solution cost.
		\end{itemize}
		
		
		The confidence level of the tests may be adjusted by using the
		option \parameter{confidence}. Increasing the value of \parameter{confidence}
		leads to a more strict statistical test.  Keep in mind that a stricter test
		will require more budget to identify which configurations perform worse. A less
		strict test discards configurations faster by requiring less evidence against
		them and, therefore, it is more likely to discard good configurations.
		
		
		\subsection{Complex parameter space constraints}
		
		Some parameters may have complex dependencies. Ideally, parameters should be
		defined in the way that is more likely to help the search performed by
		\irace.  For example, when tuning a branch and bound algorithm, one may have the
		following parameters:
		
		\begin{itemize}
			\item branching (\code{b}) that takes values in \code{\{0,1,2,3\}}, where 0 indicates no branching will be used
			and the rest are different types of branching.
			\item stabilization (\code{s}) that takes values in \code{\{0,1,2,3,4,5,6,7,8,9,10\}}, of which for \code{b=0}
			only \code{\{0,1,2,3,4,5\}} are relevant.
		\end{itemize}
		
		In this case, it is not possible to describe the parameter space by defining
		only two parameters for \irace. An extra parameter must be introduced as
		follows:
		
		\begin{center}
			\begin{minipage}{0.8\linewidth}
				\begin{CodeInput}
					# name    label   type   range                    condition
					b         "-b "   c      (0,1,2,3)
					s1        "-s "   c      (0,1,2,3,4,5)            | b == "0"
					s2        "-s "   c      (0,1,2,3,4,5,6,7,8,9,10) | b != "0"
				\end{CodeInput}
			\end{minipage}
		\end{center}
		
		Parameters whose values depend on the value of other parameters may also
		require using extra parameters or changing the parameters and processing them
		in \parameter{targetRunner}. For example, given the following parameters:
		%
		\begin{itemize}
			\item Population size (\code{p}) takes the integer values $[1,100]$.
			\item Selection size (\code{s}) takes the same values but no more than the
			population size, that is $[1,$\code{p}$]$.
		\end{itemize}
		
		In this case, it is possible to describe the parameters \code{p} and \code{s}
		using surrogate parameters for \irace that represent a ratio of the original
		interval as follows:
		
		\begin{center}
			\begin{minipage}{0.8\linewidth}
				\begin{CodeInput}
					# name    label   type   range
					p        "-p "    i      (1,100)
					s_f      "-s "   r      (0.0,1.0)
				\end{CodeInput}
			\end{minipage}
		\end{center}
		%
		and \parameter{targetRunner} must calculate the actual value of \code{s} as $\min(\max(\text{round}(\text{\code{s\_f}}\cdot \text{\code{p}}, 1)), 100)$. For
		example, if the parameter \code{p}  has value 50 and the surrogate parameter \code{s\_f} has value 0.3, then \code{s} will have value 15.
		
		The processing within \parameter{targetRunner} can also split and join
		parameters. For example, assume the following parameters:
		
		\begin{center}
			\begin{minipage}{0.8\linewidth}
				\begin{CodeInput}
					# name    label   type   range
					m         "-m "   i      (1,250)
					e         "-e "   r      (0.0,2.0)
				\end{CodeInput}
			\end{minipage}
		\end{center}
		
		These parameters could be used to define a value
		$\texttt{m}\cdot 10^\texttt{e}$ for another parameter
		(\code{--strength}) not known by \irace. Then, \parameter{targetRunner} takes care of parsing
		\code{-m} and \code{-e}, computing the strength value and passing the parameter
		\code{--strength} together with its value to the target algorithm.
		
		More complex parameter space constraints may be implemented by means of the
		\parameter{repairConfiguration} function (\autoref{sec:repairconf}).
		
		
		\subsection{Unreliable target algorithms and immediate rejection}\label{sec:reject}
		
		There are some situations in which the target algorithm may fail to execute
		correctly.  By default, \irace stops as soon as a call
		to \parameter{targetRunner} or \parameter{targetEvaluator} fails, which helps to
		detect bugs in the target algorithm.  Sometimes the failure cannot be fixed
		because it is due to system problems, network issues, memory limits, bugs for
		which no fix is available, or fixing them is impossible because there is no
		access to the source code.
		
		In those cases, if the failure is caused by random errors or transient system
		problems, one may wish to ignore the error and try again the same call in the
		hope that it succeeds. The option \parameter{targetRunnerRetries} indicates the
		number of times a \parameter{targetRunner} execution is repeated if it
		fails. Use this option only if you know additional repetitions could be
		successful.
		
		If the target algorithm consistently fails for a particular set of
		configurations, these configurations may be declared as forbidden
		(\parameter{forbiddenFile}) so that \irace avoids them.  On the other hand, if
		the configurations that cause the problem are unknown,
		the \parameter{targetRunner} should return \code{Inf} so that \irace
		immediately rejects the failing configuration. This immediate rejection should
		be used with care according to the goals of the tuning. For example, a
		configuration that crashes on a particular instance, \eg by running out of
		memory, might still be considered acceptable if it gives very good results on
		other instances. The configurations which were rejected during the execution of \irace are saved
		in the Rdata output file (see \autoref{sec:output r}).
		\begin{xwarningbox}
			If the configuration budget is specified in total execution time (\parameter{maxTime} option), 
			immediate rejected executions must provide the cost and time (which must be \code{Inf 0}). 
			Nevertheless, rejected configurations will be  excluded from the execution time estimation 
			and the execution bound calculation.
		\end{xwarningbox}
		
		\subsection{Ablation Analysis}\label{sec:ablation}
		
		The ablation method~\cite{FawHoos2015ablation} takes two configurations (source
		and target) and generates a sequence of configurations that differ between each
		other just in one parameter, where parameter values in source are replaced by
		values from target. The sequence can be seen as a ``path'' from the source to
		the target configuration.  This can be used to find new better ``intermediate''
		configurations or to analyse the impact of the parameters in the performance.
		To perform ablation use the \code{ablation} function and specify the IDs of the
		source and target configurations.  By default, the source is taken as the first
		configuration evaluated by \irace and the target as the best overall
		configuration found. The argument \code{ab.params} can be used to specify a
		subset of the parameters considered in the ablation. The option
		\parameter{firstTest} defines how many instances are selected for the
		evaluation of configurations, if a different number of instances is required it
		must be specified in the argument \code{n.instance}. If a PDF filename is
		provided (\code{pdf.file}), a plot will be produced from the ablation results
		(Fig.~\ref{fig:testAb}).
		
		<<ablation, prompt=FALSE, eval=FALSE>>=
		ablation(iraceLogFile = "irace.Rdata", 
		src = 1, target = 60, pdf.file = "plot-ablation.pdf")
		@
		
		<<testAb, fig.pos="htb!", fig.align="center", out.width="0.75\\textwidth", fig.cap="Example of plot generated by \\code{ablation()}.", prompt=FALSE, eval=TRUE, echo=FALSE>>=
		plotAblation(abLogFile = "log-ablation.Rdata")
		@
		
		The function returns a list containing the following elements:
		%
		\begin{description}
			\item \code{configurations}: A dataframe of configurations tested during ablation.
			\item \code{instances}: The instances used for the ablation.
			\item \code{scenario}: Scenario options provided by the user.
			\item \code{trajectory}: Best configuration IDs at each step of the ablation.
			\item \code{best}: Best overall configuration found.
		\end{description}
		
		
		\subsection{Postselection race}\label{sec:postselection}
		
		After the configuration process is finished it is possible perform a
		postselection race by specifying the \irace option \parameter{postselection}
		with value larger than $0$.  This option will perform a post-selection race of
		the set of best configurations of each iteration. The budget assigned for this
		race is obtained using the \parameter{postselection} option which defines a
		percentage of the \irace configuration budget. This budget is not considered in
		the total configuration budget that is, these evaluations are extra
		computation.
		
		The execution of the postselection race add an element (\code{psrace.log}) to 
		the \code{iraceResults} list saved in the \irace log file. The postselection log
		consists of a list with the following elements:
		\begin{description}
			\item \code{configurations}: Configurations used in the postselection race.
			\item \code{instances}: Instances used in the in the postselection race.
			\item \code{maxExperiments}: Configuration budget assigned for the postselection race.
			\item \code{experiments}:  Matrix of experiments in the same format as the
			\code{iraceResults$experiments} matrix. The column names indicate the candidate
			configuration identifier and the row names contain the name of the instances.
			\item \code{elites}: Elite configurations obtained in the postselection race.
		\end{description}
		
		Optionally, it is possible to perform a postselection race with all 
		elite configurations of the iterations or selecting a set of configurations
		from \code{iraceResults$allConfigurations}.
		<<postsel, prompt=FALSE, eval=FALSE>>=
		# Execute all elite configurations in the iterations
		psRace(iraceLogFile="irace.Rdata", elites=TRUE)
		# Execute a set of configurations IDs providing budget
		psRace(iraceLogFile="irace.Rdata", 
		conf.ids=c(34, 87, 102, 172, 293), 
		max.experiments=500)
		@
		
		\subsection[Parameter importance analysis using PyImp]{Parameter importance analysis using \PyImp}
		
		The \PyImp\footnote{\url{https://github.com/automl/ParameterImportance}} tool developed by the AutoML group\footnote{\url{https://www.automl.org/}}
		supports various parameter importance analysis methods using surrogate models. 
		Given a performance dataset of an algorithm configuration scenario, a Random Forest is built to predict performance of all algorithm configurations. 
		Parameter importance analyses are then applied on the prediction model. 
		The model serves as a surrogate for the original target algorithm, so that the algorithm does not need to be executed during the analyses.
		Three analysis methods are supported, namely fANOVA~\citep{HutHooLey2014icml} (functional analysis of variance), forward selection~\cite{HutHooLey2013lion},
		and ablation analysis with surrogates~\cite{BieLinEggFraFawHoo2017}. 
		Note that the \irace package directly supports ablation (without surrogate models) analysis with and without racing (\autoref{sec:ablation}).
		Although ablation analysis without surrogates may be more time-consuming, results of the surrogate version may be less accurate than the non-surrogate one.
		
		The \code{.Rdata} dataset generated by \irace can be used as input for \PyImp. 
		We provide a script to translate an \code{irace.Rdata} file into the input format required by \PyImp.
		The script is available in the \irace package, and can be accessed either through the \aR console (function \code{irace2pyimp}), or via command line (\IRACEHOME{/bin/irace2pyimp}).
		
		To see the usage of the executable, please run: \code{irace2pyimp --help}. 
		For more information on the \aR function \code{irace2pyimp}, type in the \aR console: \code{?irace2pyimp}.
		
		Given as input an \code{irace.Rdata} file, the script will generate the following output files:
		\begin{itemize}
			\item \code{params.pcs}: a text file containing the parameter space definition.
			\item \code{runhistory.json}: a JSON file containing the list of algorithm configurations evaluated during the tuning and the performance data obtained.
			\item \code{traj_aclib2.json}: a JSON file containing the best configurations after each iteration of \irace. The last configuration will be used as the target configuration in ablation analysis.
			\item \code{scenario.txt}: a text file containing the definition of the tuning scenario.
			\item \code{instances.txt}: a text file containing the list of instances.
			\item \code{features.csv}: a .csv file containing instance features. If no instance features are provided, the index of each instance will be used as a feature.
		\end{itemize}
		
		\PyImp can then be called using the files listed above as input. Several examples on how to use the script and to call \PyImp can be found at \IRACEHOME{/inst/examples/irace2pyimp/}.
		
		%\emph{\PyImp installation note:} At the time when this document is written (December 2019), the latest \PyImp release version (1.0.6) has some bugs. 
		%Please check on \url{https://pypi.org/project/PyImp/#history} to see if \PyImp version > 1.0.6 has been released. 
		%If it has, you can easily install \PyImp using \code{pip install pyimp}. 
		%Otherwise, please install \PyImp from source (\url{https://github.com/automl/ParameterImportance}), branch \code{development}. Installation instructions can be found at \url{https://automl.github.io/ParameterImportance/installation.html}.
		%%
		%% irace options
		%%
		
		\section{List of command-line and scenario options} \label{sec:irace options}
		
		Most \irace options can be specified in the command line using a flag or in the
		\irace scenario file using the option name (or setting their value in the
		\code{scenario} list passed to the various \aR functions exported by the
		package). This section describes the various \irace options that can be
		specified by the user in this way.
		
		\begin{xwarningbox}
			Relative filesystem paths (e.g., \path{../scenario/}) given in the
			command-line are relative to the current working directory (the directory at
			which \irace is invoked). However, paths given in
			the scenario file are relative to the directory containing
			the scenario file. See also Table~\ref{tab:paths}.
		\end{xwarningbox}
		
		\input{section/irace-options}
		
		%%
		%%
		%%
		%% FAQ
		%%
		%%
		%%
		
		
		\section{FAQ (Frequently Asked Questions)}
		
		\subsection{Is \irace minimizing or maximizing the output of my algorithm?}
		
		By default, \irace considers that the value returned by \parameter{targetRunner} (or by
		\parameter{targetEvaluator}, if used) should be \underline{\textbf{minimized}}. In case of a maximization
		problem, one can simply multiply the value by -1 before returning it to
		irace. This is done, for example, when maximizing the hypervolume (see the last
		lines in \IRACEHOME{/examples/hypervolume/target-evaluator}).
		
		
		\subsection[Is it possible to configure a MATLAB algorithm with irace?]
		{Is it possible to configure a \MATLAB algorithm with \irace ?}
		
		Definitely. There are three main ways to achieve this:
		\begin{enumerate}
			\item Edit the \parameter{targetRunner} script to call \MATLAB in a non-interactive
			way. See the \MATLAB documentation, or the following links.\footnote{\url{http://stackoverflow.com/questions/1518072/suppress-start-message-of-matlab}}\footnote{\url{http://stackoverflow.com/questions/4611195/how-to-call-matlab-from-command-line-and-print-to-stdout-before-exiting}} %
			You would need to pass the parameter received by \parameter{targetRunner} to your \MATLAB script.\footnote{\url{https://www.mathworks.com/matlabcentral/answers/97204-how-can-i-pass-input-parameters-when-running-matlab-in-batch-mode-in-windows}}\footnote{\url{https://stackoverflow.com/questions/3335505/how-can-i-pass-command-line-arguments-to-a-standalone-matlab-executable-running}} 
			%
			There is a minimal example in \IRACEHOME{/examples/matlab/}.
			
			\item Call \MATLAB code directly from \aR using the
			\pkg{matlabr} package (\url{https://cran.r-project.org/package=matlabr}). This
			is a better option if you are experienced in \aR. Define \parameter{targetRunner} as
			an \aR function instead of a path to a script. The function should
			call your \MATLAB code with appropriate parameters.
			
			\item Another possibility is calling \MATLAB directly from a different programming language and write \parameter{targetRunner} in that programming language,  for example, in \proglang{Python} (see examples in \IRACEHOME{/examples/target-runner-python/}).\footnote{\url{https://www.mathworks.com/help/matlab/matlab_external/call-matlab-functions-from-python.html}\\
				\url{https://www.mathworks.com/help/matlab/matlab_external/call-user-script-and-function-from-python.html}}
		\end{enumerate}
		
		\subsection{My program works perfectly on its own, but not when running under \irace. Is irace broken?}\label{faq:valgrind}
		
		Every time this was reported, it was a difficult-to-reproduce bug, i.e., a \href{https://en.wikipedia.org/wiki/Heisenbug}{Heisenbug}, in
		the program (target algorithm), not in \irace.
		%
		To detect such bugs, we recommend that you use, within
		\parameter{targetRunner}, a memory debugger (e.g.,
		\href{http://valgrind.org/}{\code{valgrind}}) to run your program. For example,
		if your program is executed by \parameter{targetRunner} as:
		% 
		\begin{lstlisting}[style=BashInputStyle]
			${EXE} ${FIXED_PARAMS} -i ${INSTANCE} ${CONFIG_PARAMS} 1> ${STDOUT} 2> ${STDERR}
		\end{lstlisting}
		%
		then replace that line with:
		%
		\begin{lstlisting}[style=BashInputStyle]
			valgrind --error-exitcode=1 ${EXE} ${FIXED_PARAMS} -i ${INSTANCE} \
			${CONFIG_PARAMS} 1> ${STDOUT} 2> ${STDERR}
		\end{lstlisting}
		
		If there are bugs in your program, they will appear in
		\code{\${STDERR}}, thus do not delete those files. Memory debuggers will
		significantly slowdown your code, so use them only as a means to find what is
		wrong with your target algorithm. Once you have fixed the bugs, you should
		remove the use of \code{valgrind}.
		
		
		\subsection{\irace seems to run forever without any progress, is this a bug?}\label{faq:infloop}
		
		Every time this problem was reported, the issue was in the target algorithm and
		not in \irace. Some ideas for debugging this problem:
		\begin{itemize}
			\item Check that the target algorithm is really not running nor paused nor
			sleeping nor waiting for input-output.
			\item Use \parameter{debugLevel}\code{=3} to see how  \irace calls \code{target-runner}, run the same command outside \irace and verify that it terminates.
			\item Add some output to your algorithm that reports at the very end the runtime and exit code. Verify that this output is printed when \irace calls your algorithm. 
			\item In \code{target-runner}, print something to a log file \emph{after} calling your target algorithm. Verify that this output appears in the log file when \irace is running.
			
			\item Set a maximum timeout when calling your target algorithm from \code{target-runner} (see FAQ~\ref{faq:timeout}).
			
		\end{itemize}
		
		\subsection{My program may be buggy and run into an infinite loop. Is it possible to set a maximum timeout?}\label{faq:timeout}
		
		We are not aware of any way to achieve this using \aR. However, in
		GNU/Linux, it is easy to implement by using the \code{timeout} command\footnote{\url{http://man7.org/linux/man-pages/man1/timeout.1.html}} in
		\code{targetRunner} when invoking your program.
		
		\subsection{When using the mpi option, \irace is aborted with an error message indicating that a
			function is not defined. How to fix this?}\label{sec:mpi-error}
		
		\pkg{Rmpi} does not work the same way when called from within a package and when called from a script or 
		interactively. When \irace creates the slave nodes, the slaves will load a copy of \irace automatically. 
		If the slave nodes are on different machines, they must have \irace installed. If \irace is not installed
		system-wide, \aR needs to be able to find \irace on the slave nodes. This is usually done by setting \code{R\_LIBS}, 
		\code{.libPaths()} or by loading \irace using \code{library()} or \code{require()} with the argument ``\code{lib.loc}''. 
		The settings on the master are not applied to the slave nodes automatically, thus the slave 
		nodes may need their own settings. After spawning the slaves, it is too late to modify 
		those settings, thus modifying the shell variable \code{R\_LIBS} seems the only valid 
		way to tell the slaves where to find \irace.
		
		If the path is set correctly and the problem persists, please check these instructions:
		% 
		\begin{enumerate}[leftmargin=*,widest=3]
			\item Test that \irace and \pkg{Rmpi} work. Run \irace on a single machine (submit node), without calling \code{qsub}, \code{mpirun} or a 
			similar wrapper around \irace or \aR.
			\item Test loading \irace on the slave nodes. However,
			jobs submitted by \code{qsub}/\code{mpirun} may load \aR packages using a different mechanism from the way it happens if you log directly into the node (e.g., with \code{ssh}). Thus, you need to write a little \aR program such as:
			<<faq3, eval=FALSE>>=
			library(Rmpi)
			mpi.spawn.Rslaves(nslaves = 10)
			paths <- mpi.applyLB(1:10, function(x) {
				library(irace); return(path.package("irace")) })
			print(paths)
			@
			Submit this program to the cluster like you would submit \irace (using \code{qsub}, \code{mpirun} or whatever program is used to submit jobs to the cluster).
			
			\item In the script \code{bin/parallel-irace-mpi}, the function \code{irace\_main()} creates an MPI job for our cluster. 
			You may need to speak with the admin of your cluster and ask them how to best submit a job for MPI. There may be 
			some particular settings that you need. \pkg{Rmpi} normally creates  log files; but \irace suppresses those files unless $\texttt{debugLevel} > 0$.
		\end{enumerate}
		
		Please contact us (\autoref{sec:contact}) if you have further problems.
		
		\subsection[Error: 4 arguments passed to {.Internal(nchar)} which requires 3]{Error: 4 arguments passed to \code{.Internal(nchar)} which requires 3}
		
		This is a bug in \aR 3.2.0 on Windows. The solution is to update your version of
		\aR.
		
		\subsection{Warning: \code{In read.table(filename,}  \code{header = TRUE,} \code{colClasses = "character",} \code{: incomplete final line found by} \ldots}
		
		This is a warning given by \aR when the last line of an input file does not
		finish with the newline character. The warning is harmless and can be
		ignored. If you want to suppress it, just open the file and press the
		\code{ENTER} key at the end of the last line of the file.
		
		
		\subsection{How are relative filesystem paths interpreted by \irace?}\label{faq:relpaths}
		
		The answer depends on where the path appears. Relative paths may appear as the
		argument of command-line options, as the value of options given in the scenario
		file, or within various scripts, functions or instance files. Table
		\ref{tab:paths} summarizes how paths are translated from relative to absolute.
		
		\begin{table}[h!]
			\centering
			\caption{Translation of relative to absolute filesystem paths.}
			\label{tab:paths}
			\begin{tabular}{rl}
				\toprule
				\bf    Relative path appears as \ldots  & \bf\ldots is relative to \ldots \\
				\midrule
				a string within \parameter{trainInstancesFile} & \parameter{trainInstancesDir}\\
				a string within \parameter{testInstancesFile} & \parameter{testInstancesDir}\\
				code within \parameter{targetRunner} or \parameter{targetEvaluator} & \parameter{execDir}\\
				the value of \parameter{logFile} or \code{--log-file} & \parameter{execDir}\\
				the value of other options in the scenario file  &  the directory containing the scenario file\\
				the value of other command-line options & invocation (working) directory of \irace\\
				\bottomrule    
			\end{tabular}
		\end{table}
		
		\subsection{My parameter space is small enough that \irace could generate all possible configurations; however, \irace generates repeated configurations and/or does not generate some of them. Is this a bug?}\label{faq:allconfs}
		
		Currently, \irace does not try to detect whether all possible configurations
		can be evaluated for the given budget, thus, the initial random sampling
		performed by \irace may generate repeated configurations and/or never generate
		some configurations, which is not ideal. The ideal approach in such cases is to
		provide all configurations explicitly to \irace (\autoref{sec:initial}) and
		execute a single race (\parameter{nbIterations}\code{=1}) with exactly the
		number of configurations provided (e.g.,
		\parameter{nbConfigurations}\code{=10}). A future version of \irace will
		automatically detect this case and switch to basic racing without having to set additional options.
		
		%%
		%%
		%%
		%% Contact info
		%%
		%%
		%%
		
		
		%FIXME: complete this section
		%\section{Known problems}
		
		\section{Resources and contact information} \label{sec:contact}
		
		More information about the package can be found on the \irace webpage:
		\begin{center} \url{http://iridia.ulb.ac.be/irace/} \end{center}
		
		For questions and suggestions please contact the development team through 
		the \irace package Google group:
		\begin{center}
			\url{https://groups.google.com/d/forum/irace-package}
		\end{center} 
		or by sending an email to:
		\begin{center}
			\href{mailto:irace-package@googlegroups.com}{irace-package@googlegroups.com}
		\end{center}
		
		%% MANUEL: I think we should use just one channel. If the Google group is the winner, that should be it.
		%the mailing list:
		%% LESLIE I agree.
		%\begin{center} \href{mailto:irace@iridia.ulb.ac.be}{irace@iridia.ulb.ac.be} \end{center}
		
		
		\section{Acknowledgements}
		
		We would like to thank all the people that directly or indirectly have
		collaborated in the development and improvement of \irace:
		%
		\newlist{acklist}{itemize*}{1}
		\setlist[acklist]{label={}, before=\unskip{}, afterlabel=\unskip{ }, itemjoin={{,}}, itemjoin*={, and }, after={{.}}}
		\begin{acklist}
			\item Prasanna Balaprakash
			\item Zhi (Eric) Yuan
			\item Franco Mascia
			\item Alberto Franzin
			\item Anthony Antoun
			\item Esteban Diaz Leiva %MATLAB example
			\item Federico Caselli % bug reports with fixes
			\item Pablo Valledor Pellicer % testing of --parallel under Windows
			\item André de Souza Andrade
			\item Nguyen Dang (\texttt{nttd@st-andrews.ac.uk}) % implemented \code{irace2pyimp}
		\end{acklist}
		
		
		\bibliographystyle{abbrvnat}
		\ifthenelse {\boolean{Release}}{%
			\bibliography{irace-package}%
		}{%
			\bibliography{optbib/abbrev,optbib/journals,optbib/authors,optbib/biblio,optbib/crossref}%
		}
		
		\newpage
		
		\begin{appendices}
			
			\section{Installing R} \label{sec:installation}
			
			This section gives a quick \aR installation guide that will work in
			most cases. The official instructions are available at
			\url{https://cran.r-project.org/doc/manuals/r-release/R-admin.html}
			
			\subsection{GNU/Linux}
			You should install \aR from your package manager. On a Debian/Ubuntu system
			it will be something like:
			\begin{lstlisting}[style=BashInputStyle]
				sudo apt-get install r-base
			\end{lstlisting}
			
			Once \aR is installed, you can launch \aR from the Terminal and from the \aR prompt
			install the \irace package (see \autoref{sec:irace install}).
			
			
			\subsection{OS X}
			
			You can install \aR directly from a CRAN mirror.\footnote{\url{https://cran.r-project.org/bin/macosx/}} %
			Alternatively, if you use
			homebrew, you can just brew the \aR formula from the science tap (unfortunately
			it does not come already bottled so you need to have Xcode\footnote{Xcode download
				webpage: \url{https://developer.apple.com/xcode/download/}} installed to compile it):
			%<<R_OS_install,engine='bash',eval=FALSE>>=
			\begin{lstlisting}[style=BashInputStyle]
				brew tap homebrew/science
				brew install r
			\end{lstlisting}
			%@
			
			Once \aR is installed, you can launch \aR from the Terminal (or from your Applications),
			and from the \aR prompt install the \irace package (see \autoref{sec:irace install}).
			
			\subsection{Windows}
			
			You can install \aR from a CRAN mirror.\footnote{\url{https://cran.r-project.org/bin/windows/}} %
			We recommend that you
			install \aR on a filesystem path without spaces, special characters or long
			names, such as \path{C:\R}. Once \aR is installed, you can launch the \aR
			console and install the \irace package from it (see \autoref{sec:irace
				install}).
			
			
			\section{targetRunner troubleshooting checklist} \label{sec:check list}
			
			If the \parameter{targetRunner} script fails to return the output expected by
			\irace, it can be sometimes difficult to diagnose where the problem lies. The
			more descriptive errors provided by your script, the easier it will be to debug
			it. If \parameter{targetRunner} enters an infinite loop, irace will wait
			indefinitely (see FAQ in \autoref{faq:timeout}).  If you are using temporary
			files to redirect the output of your algorithm, check that these files are
			properly created. We recommend to follow the structure of the example file
			(\code{target-runner}) provided in \IRACEHOME{/templates}. The following
			error examples are based on that example file.
			
			In case of failure of \parameter{targetRunner}, \irace will print an error on
			its output describing which execution of \parameter{targetRunner} failed.
			%
			Follow this checklist to detect where the problem is:
			
			\begin{enumerate}[leftmargin=*,widest=9]
				
				\item Make sure that your \code{targetRunner} script or program is at the specified location. If you see this error:
				\begin{CodeInput}
					Error: == irace == target runner '~/tuning/target-runner' does not exist
				\end{CodeInput}
				%
				it means that \irace cannot find the \code{target-runner} file. Check that the file is at
				the path specified by the error.
				
				\item Make sure that your \code{targetRunner} script is an executable file and the user running \irace has permission to execute it. The following errors:
				%
				\begin{CodeInput}
					Error: == irace == target runner '~/tuning/target-runner' is a directory,
					not a file
				\end{CodeInput}
				or
				\begin{CodeInput}
					Error: == irace == target runner '~/tuning/target-runner' is not executable
				\end{CodeInput}
				%
				mean that your \code{targetRunner} is not an executable file. In the first
				case, the script is a folder and therefore there must be a problem with the
				name of the script. In the second case, you must make the file executable,
				which in GNU/Linux can be done by:
				%
				\begin{lstlisting}[style=BashInputStyle]
					chmod +x ~/tuning/target-runner
				\end{lstlisting}
				
				\item If your \code{targetRunner} script calls another program, make sure it is at  the location described in the script
				(variable \code{EXE} in the examples and templates). A typical output for such an error is:
				%
				\begin{CodeInput}
					Error: == irace == running command ''~/tuning/target-runner' 1 8 676651103 
					~/tuning/Instances/1000-16.tsp --ras --localsearch 2 --alpha 4.03 --beta 1.89
					--rho  0.02 --ants 37 --nnls 48 --dlb 0 --rasranks 15 2>\&1' had status 1  
					== irace == The call to target.runner.default was:
					~/tuning/target-runner 1 8 676651103 ~/tuning/Instances/1000-16.tsp --ras 
					--localsearch 2 --alpha 4.03 --beta 1.89 --rho  0.02 --ants 37 --nnls 48 
					--dlb 0 --rasranks 15 
					== irace == The output was:
					Tue May  3 19:00:37 UTC 2016: error: ~/bin/acotsp: not found or not executable
					(pwd: ~/tuning/acotsp-arena)
				\end{CodeInput}
				%
				You may test your script by copying the command line shown in the error and
				executing \code{target-runner} directly on the execution directory (\parameter{execDir}). In this case, the command line is:
				%
				\begin{lstlisting}[style=BashInputStyle]
					~/tuning/target-runner 1 8 676651103 ~/tuning/Instances/1000-16.tsp  --ras \
					--localsearch 2 --alpha 4.03 --beta 1.89 --rho  0.02 --ants 37 --nnls 48 \
					--dlb 0 --rasranks 15
				\end{lstlisting}
				
				This executes the \code{targetRunner} script as \irace does. The output of this script must be only one number. 
				
				\item Check that your \code{targetRunner} script is actually returning one
				number as output. For example:
				% 
				\begin{CodeInput}
					Error: == irace == The output of '~/tuning/target-runner 1 25 365157769
					~/tuning/Instances/1000-31.tsp --ras --localsearch 1 --alpha 0.26 --beta
					6.95 --rho 0.69 --ants 56 --nnls 10 --dlb 0 --rasranks 7' is not numeric! 
					== irace == The output was:
					Solution: 24479793
				\end{CodeInput}
				% FIXME: Duplicated above.
				%% For testing your script, copy the command-line of \code{target-runner} and execute it directly on the execution directory (\parameter{execDir}):
				%% %
				%% \begin{lstlisting}[style=BashInputStyle]
					%% ~/tuning/target-runner 1 25 365157769 ~/tuning/Instances/1000-31.tsp  --ras \
					%%  --localsearch 1 --alpha 0.26 --beta 6.95 --rho  0.69 --ants 56 \
					%%  --nnls 10 --dlb 0 --rasranks 7
					%% \end{lstlisting}
				%% This executes the \code{targetRunner} script as \irace does. 
				In the example above, the output of \code{target-runner} is ``\code{Solution:
					24479793}'', which is not a number. If \code{target-runner} is parsing the
				output of the target algorithm, you need to verify that the code only parses
				the solution cost value.
				
				\item Check that your \code{targetRunner} script is creating the output files
				for your algorithm.  If you see an error as:
				%
				\begin{CodeInput}
					== irace == The output was: Tue May  3 19:41:40 UTC 2016: 
					error: c1-9.stdout: No such file or directory
				\end{CodeInput}
				The output file of the execution of your algorithm has not been
				created (check permissions) or has been deleted before the result can be read.
				
				
				\item Other errors can produce the following output:
				%
				\begin{CodeInput}
					== irace == The output was: Tue May  3 19:49:06 UTC 2016: 
					error: c1-23.stdout: Output is not a number
				\end{CodeInput}
				%
				This might be because your \code{targetRunner} script is not executing your
				algorithm correctly. To further investigate this issue, comment out the line
				that eliminates the temporary files that saves the output of your
				algorithm. Similar to this one
				%
				\begin{lstlisting}[style=BashInputStyle]
					rm -f "${STDOUT}" "${STDERR}"
				\end{lstlisting}
				%
				Execute directly the \code{targetRunner} command-line that is provided in the
				error message, look in your execution directory for the files that are
				created. Check the \code{.stderr} file for errors and the \code{.stdout} file
				to see the output that your algorithm produces.
				
				\item Some command within \code{targetRunner} may not be working correctly. In
				that case, you must debug the commands individually exactly as \irace executes them. In
				order to find where the problem is, print the commands to a log file before
				executing them. For example:
				% 
				\begin{lstlisting}[style=BashInputStyle]
					echo "$EXE ${FIXED_PARAMS} -i $INSTANCE ${CONFIG_PARAMS}" >> ${STDERR}.log
					$EXE ${FIXED_PARAMS} -i $INSTANCE ${CONFIG_PARAMS} 1> ${STDOUT} 2> ${STDERR}
				\end{lstlisting}
				%
				then look at the \code{${STDERR}.log} file corresponding to the
				\code{targetRunner} call that failed and execute/debug the last command
				there.
				
				\item If the language of your operating system, the \code{target-runner} or the
				target algorithm is not English, \irace may not be able to recognize the
				numbers generated by \code{target-runner}. We recommend that you run \irace,
				the \code{target-runner} and the target algorithm under an English locale (or
				make sure that their languages and number format are compatible).
				
				\item It is possible that
				\href{https://en.wikipedia.org/wiki/Heisenbug}{transient bugs} in the target
				algorithm are only visible when running within \irace, and
				all commands within \code{targetRunner} appear to work fine when executed directly in the
				command-line outside \irace. See FAQ in \autoref{faq:valgrind}) for suggestions on how to detect such bugs.
				
				
				\item If your \code{targetRunner} script works when running irace with \parameter{parallel}\code{=0} but it fails when using higher number of cores, this may be due to any number of reasons:
				\begin{itemize}
					\item If you submit jobs through a queuing system, the running environment
					when using the queuing system may not be the same as when you launch \irace
					yourself. The queuing system may also send the job to different machines
					depending on the number of CPUs requested. One way to test this is to
					submit the failing execution of \code{targetRunner} to the queuing system,
					and specifically to any problematic machine.
					
					\item When using MPI, some calls to \code{targetRunner} may run on different
					computers than the one running the master \irace process. See FAQ in
					\autoref{sec:mpi-error}.
					
					\item Does \code{targetRunner} read or create intermediate files? These files
					may cause a race condition when two calls to \code{targetRunner} happen at
					the same time. You have to make sure that parallel runs of
					\code{targetRunner} do not interfere with each other's files.
					
					\item Maybe these files consume too much memory or fill the filesystem when
					there are simultaneous \code{targetRunner} calls? Moreover, queuing systems
					have stricter limits for computing nodes than for the submit/host node.
					
					\item Does the machine or the queuing system impose any limits on number of
					processes or CPU/memory/filesystem usage per job? Such limits may only
					trigger when more than one process is executed in parallel, killing the
					\code{targetRunner} process before it has a chance to print anything
					useful. In that case, \irace may not detect the the program finished
					unexpectedly, only that the expected output was not printed.
				\end{itemize}
				
			\end{enumerate}
			
			\section{targetEvaluator troubleshooting checklist} \label{sec:evaluator check list}
			
			Even if \parameter{targetRunner} appears to work, the use of \parameter{targetEvaluator} may lead to other problems. The same checklist of \code{targetRunner} can be followed here. In addition, we list here other potential problems unique to \code{targetEvaluator}:
			
			\begin{enumerate}
				\item If \parameter{targetEvaluator} fails only in the second or later
				iteration, this may because output files or data generated by a previous call
				to \parameter{targetRunner} are missing. Elite configurations are never
				re-executed on the same instance and seed pair, that is, \irace will call
				only once \parameter{targetRunner} for each pair of configuration ID and
				instance ID. However, \parameter{targetEvaluator} is always re-executed,
				which takes into account any updated information (normalization bounds,
				reference sets/points, best-known values, etc.). Thus, any files or data
				generated by \parameter{targetRunner} for a given configuration must remain
				available to \parameter{targetEvaluator} as long as that configuration is
				alive. The list of alive configurations is passed
				to \parameter{targetEvaluator}, which may decide then which data to keep or
				remove.
			\end{enumerate}
			
			\section{Glossary}
			
			\begin{description}
				\item[Parameter tuning:] Process of searching good settings for the parameters
				of an algorithm under a particular tuning scenario (instances, execution
				time, etc.).
				\item[Scenario:] Settings that define an instance of the tuning problem. These
				settings include the algorithm to be tuned (target), budget for the execution
				of the target algorithm (execution time, evaluations, iterations, etc.), set
				of problem instances and all the information that is required to perform the
				tuning.
				\item[Target algorithm:] Algorithm whose parameters will be tuned.
				\item[Target parameter:] Parameter of the target algorithm that will be tuned.
				\item[\irace option:] Configurable option of \irace.
				\item[Elite configurations:] Best configurations found so far by \irace. New
				configurations for the next iteration of \irace are sampled from the
				probabilistic models associated to the elite configurations.  All elite
				configurations are also included in the next iteration.
				\item[\hypertarget{irace_home}{\path{$IRACE_HOME}:}] The filesystem path where \irace is
				installed. You can find this information by opening an \aR console and
				executing:
				
				<<R_irace_home2, prompt=FALSE, eval=FALSE>>=
				system.file(package = "irace")
				@
				
			\end{description}
			
			\section{NEWS}
			
			\RecustomVerbatimCommand{\VerbatimInput}{VerbatimInput}%
			{fontsize=\footnotesize,
				%
				frame=lines,  % top and bottom rule only
				framesep=1em, % separation between frame and text
				rulecolor=\color{Gray},
				%
				label=\fbox{\color{Black}NEWS},
				labelposition=topline,
				%
				% commandchars=\|\(\), % escape character and argument delimiters for
				%                      % commands within the verbatim
				% commentchar=*        % comment character
			}
			
			%% NEWS is latin1 but this file is utf8
			\begingroup\inputencoding{utf8}%
			\VerbatimInput{NEWS.txt}%
			\endgroup
			
		\end{appendices}
		
	\end{document}
	
	%%% Local Variables:
	%%% TeX-master: "irace-package.Rnw"
	%%% End:
	
	% LocalWords: iteratively