% !TeX spellcheck = en_GB
\documentclass{article}
\usepackage[latin1]{inputenc}
\usepackage{textcomp}
\usepackage[T1]{fontenc}
\usepackage{hyperref}
\usepackage{color, colortbl}

\usepackage{natbib}
%\bibliographystyle{elsarticle-harv}
\bibliographystyle{apalike} 

\usepackage{graphicx}
\usepackage{float} % to position figures and tables

\newcommand{\boldm}[1] {\mathversion{bold}#1\mathversion{normal}}

\usepackage{booktabs}

\usepackage{tikz}
\usetikzlibrary{calc}
\usetikzlibrary{positioning}
\usetikzlibrary{shapes.arrows}
% lets get the exact color
\usepackage{xcolor}
\definecolor{myblue}{HTML}{B0D7FF}

\tikzset{
	mybox/.style={rectangle,
		draw,
		fill= myblue,
		rounded corners,
		minimum width=2cm,
		inner sep=5pt,
		align=left,
		minimum height=1cm
	},
	myarrow/.style={draw=black,
		fill=white,
		minimum width=0.6cm,
		single arrow
	},
	longarrow/.style={draw=none,
		shading=axis,
		left color=white,
		right color=myblue,
		minimum width=0.6cm,
		single arrow,
		anchor=east
	}
}


\begin{document}
\thispagestyle{empty}


\begin{center}


{\Large\bf Air cargo load and route planning\\ in pickup and delivery operations}\\
\vspace{0.5cm}
{\em Antonio Celio Pereira de Mesquita \\ Carlos Alberto Alonso Sanches}
\end{center}



The authors express their gratitude to the three anonymous reviewers for their comments and criticism, which helped to improve the content and presentation of this paper. We carefully considered all comments from the reviewers, and proceeded to the revision of the text.

We do understand that now our paper achieves the high standard quality of this journal. As will be presented below, we followed the requests from reviewers.


\section{\#1-a: The entire editorial decision letter}

\footnotesize

Ms. Ref. No.: ESWA-D-23-16263\\
Title: Air cargo load and route planning in pickup and delivery operations \\
Expert Systems With Applications \\

Dear Carlos,\\

As Editor, I'm writing this editorial decision letter on your paper submission.  If you are interested in submitting a revised version, please read through this entire editorial decision letter carefully and take all actions seriously in order to avoid any delay in the review process of your revised manuscript submission. You need to upload a 'Detailed Response to Reviewers' in the EM system with the following sections while submitting the revised manuscript. Please note that the Required Sections (Section \#1-a, Section \#1-b, Section \#2-a, Section \#2-b, Section \#3-a, and Section \#3-b) defined clearly in this editorial decision letter with specific Compliance Requirements (part of this editorial decision letter) must be clearly labeled and included in the 'Detailed Response to Reviewers'.  The Required Sections must be clearly placed before the revised manuscript. NOTE: Section \#1-(a) MUST contain the COMPLETE text covered in this email letter from Editor-in-Chief, rather than just only the first paragraph of this email letter from Editor-in-Chief. Please note that we will NOT admit any revised manuscript submission for further review if any of the Required Sections is incomplete or any non-compliance of the ESWA authors' guidelines in the PDF file of the revised manuscript that you approve in EM system. So you need to take both (1) Required Sections and (2) Compliance Requirements seriously to avoid any delay in the review process of your revised manuscript. The font size of the Required Sections should be consistent and readable. All required sections should not be embedded. The Required Sections must be placed clearly BEFORE the revised manuscript and in the order listed as followings:

REQUIRED SECTIONS: Please confirm that your submission includes ALL of  the required sections.

\begin{itemize}

\item Section \#1-a: Including the entire editorial decision letter (i.e., complete text covered in this email letter) from Editor

\item Section \#1-b: Including your responses to Editor

\item Section \#2-a: Including the entire comments made by the Associate Editor

\item Section \#2-b: Including your Point-to-Point responses to the Associate Editor

\item Section \#3-a: Including the entire comments made by the Reviewer

\item Section \#3-b. Including your Point-to-Point responses to the Reviewer.

\end{itemize}

COMPLIANCE REQUIREMENTS:  Please confirm that your submission completely complies with the compliance requirements.

Please note that prior to admitting the revised submission to the next rigorous review process, all paper submissions must completely comply with ESWA Guide for Authors (see details at \url{https://www.elsevier.com/journals/expert-systems-with-applications/0957-4174/guide-for-authors}).  These include at least the following Compliance Requirements:

A) Authorship policies - Please also note that ESWA takes authorship very seriously and all paper submissions MUST completely comply with all of the following three policies on authorship (clearly stated in the questionnaire responses in EM system) prior to a rigorous peer review process:

A)-1: The corresponding author needs to enter the full names, full affiliation with country and email address of every contributing author in EM online system.  It is also mandatory that every contributing coauthor must be listed in EM at submission.

A)-2: It is mandatory that the full names, full affiliation with country and email address of every contributing author must be included in title (authorship) page of the manuscript. The first page of the manuscript should contain the title of the paper, and the full name, full affiliation with country and email address of every contributing author. The second page of the manuscript should begin with the paper abstract. Note that cover letter is not title (authorship) page. 

A)-3: The authorship information in EM system must be consistent with the authorship information on the title (authorship) page of the manuscript.

B)  Guidelines of reference style and reference list - Citations in the text should follow the referencing style used by the American Psychological Association (APA).

B)-1: Reference Style: Citations in the text should follow the referencing style used by the American Psychological Association. You are referred to the Publication Manual of the American Psychological Association, Sixth Edition, ISBN 978-1-4338-0561-5.  APA's in-text citations require the author's last name and the year of publication. You should cite publications in the text, for example, (Smith, 2020).  However, you should not use [Smith, 2020]. Note: There should be no [1], [2], [3], etc in your manuscript.

B)-2: Reference List:

References should be arranged first alphabetically by the surname of the first author followed by initials of the author's given name, and then further sorted chronologically if necessary. More than one reference from the same author(s) in the same year must be identified by the letters 'a', 'b', 'c', etc., placed after the year of publication. For example, Van der Geer, J., Hanraads, J. A. J., \& Lupton, R. A. (2010). The art of writing a scientific article. Journal of Scientific Communications, 163, 51-59. \url{https://doi.org/10.1016/j.Sc.2010.00372}. Note: There should be no [1], [2], [3], etc in your references list.

C) Highlights guidelines - There should be a maximum of 85 characters, including spaces, per Highlight in the Highlights section. Please kindly read this guideline carefully - the guideline does NOT say there should be a maximum of 85 words per Highlight.  It says there should be a maximum of 85 characters per Highlight.  As examples, the word "impact" consists of 6 characters; the word "significance" consists of 12 characters. Only include 3 to 5 Highlights. Minimum number is 3, and maximum number is 5.

D) Please pay attention to the new journal policy asking for institutional emails in all submissions. Every author must have an institutional email in both, the submission platform and manuscript.

NOTE: Your paper submission will be returned to authors and will NOT be admitted to further review if the revised paper fails to completely comply with the ESWA Authorship policies, ESWA guidelines of reference style and reference list, Highlights guidelines. You need to take the Compliance Requirements seriously to avoid any delay in the review process of your revised manuscript.

To submit a Complete and Compliance revision, please go to \url{https://www.editorialmanager.com/eswa/} and login as an Author.a


Your username is: \url{alonso@ita.br}

If you need to retrieve password details, please go to:
click here to reset your password    

On your Main Menu page is a folder entitled "Submissions Needing Revision". You will find your submission record there.

When you are ready to submit your revised paper, please upload the source files of your revised paper (if in Word, please upload the .doc or docx file; if in .TeX, please upload the .TeX, style) as these would be required in Production if your manuscript gets accepted

Please note that we allow 21 days for the first author revision and 14 days for any additional author revisions that are required.

Note: While submitting the revised manuscript, please double check the author names provided in the submission so that authorship related changes are made in the revision stage. If your manuscript is accepted, any authorship change will involve approval from co-authors and respective editor handling the submission and this may cause a significant delay in publishing your manuscript.Please also check which journals do you select for publication. There is no change back under the condition that the manuscript is accepted

Research Elements (optional)
This journal encourages you to share research objects - including your raw data, methods, protocols, software, hardware and more - which support your original research article in a Research Elements journal. Research Elements are open access, multidisciplinary, peer-reviewed journals which make the objects associated with your research more discoverable, trustworthy and promote replicability and reproducibility. As open access journals, there may be an Article Publishing Charge if your paper is accepted for publication. Find out more about the Research Elements journals at
\url{https://www.elsevier.com/authors/tools-and-resources/research-elements-journals?dgcid=ec_em_research_elements_email}.\\

Yours sincerely,\\

Joaqu�n Torres-Sospedra, Ph.D.

Editor

Expert Systems With Applications

\normalsize


\section{\#1-b: Our responses to Editor}

We appreciate and comply with all the Editor's instructions::

\begin{itemize}
\item We have included all {\bf Required Sections} in this text, responding individually to each of the reviewers' comments.

\item Submission of this revised article meets all {\bf Compliance Requirements}. Regarding {\it Reference Style}, we use {\tt bibliographystyle\{apalike\}},\\ available in \LaTeX.

\end{itemize}


\section{\#2-a: The entire comments made by the Associate Editor}

According to the message received, it appears that the Associate Editor's comments are as follows:

\vspace{0.5cm}


Reviewers' comments:

\begin{itemize}

\item [1.] More analysis shall be given to discuss the experimental results.
\item [2.] More newly-published related references shall be cited and discussed in Introduction.
\item [3.] More experiments hope to be added to support the conclusion.
\end{itemize}


\section{\#2-b: Our Point-to-Point responses to the Associate Editor}


\paragraph{Revision:}
More analysis shall be given to discuss the experimental results. 
\begin{center}
	\begin{minipage}{10cm}
		\textit{As we substantially increased the experimental data presented in the article (see the third answer in this section), there were also greater discussions about the results obtained. In any case, we emphasise that we are willing to clarify any result that is not sufficiently discussed; to do this, reviewers simply need to explicitly indicate what they are about.}
	\end{minipage}
\end{center}



\paragraph{Revision:}
More newly-published related references shall be cited and discussed in Introduction.
\begin{center}
	\begin{minipage}{10cm}
		\textit{In Section 2, we highlight the most recent references and concisely discuss their main aspects. To the best of our knowledge, we have not failed to cite any recent and relevant publications related to the topic of this work. This can be confirmed through the bibliography of \citep{zhao2023}, published less than a year ago.}
		
	\end{minipage}
\end{center}



\paragraph{Revision:}
More experiments hope to be added to support the conclusion.
\begin{center}
	\begin{minipage}{10cm}
		\textit{We substantially increased the experimental data: practical determination of Shims parameters (Table 10); distribution of the runtime among nodes in proportion to the volume available for shipment; performance comparison of five meta-heuristics in the node-by-node solution (Table 11); application and results of Shims in contexts with a greater number of nodes (subsubsection 6.2.2). If reviewers request any more specific experiments, we are willing to perform them.}

	\end{minipage}
\end{center}

\section{\#3-a: The entire comments made by the Reviewers}

We indicate below, in each subsection, the comments made by the three reviewers.

\subsection{Reviewer \#1}

\begin{itemize}

\item [1] In section 2, Literature review in the first and second paragraphs is too simple. Some analysis needs to be given to introduce the research status.

\item [2] In the manuscript, any figures about algorithm simulation results cannot be found. Some figures and analysis are suggested to be presented to show the superiority of the used algorithm.

\item [3] Also, any figures about experiments are not given in the manuscript. "By using a portable computer, our strategy quickly found practical solutions to a wide range of real problems in much less than operationally acceptable time."
Could you please show the results with some figures?

\item [4] Too many old references are cited.

\end{itemize}

\subsection{Reviewer \#2}

This paper addresses the optimization of air cargo load planning and routing for pickup and delivery operations. It introduces a novel problem called the Air Cargo Load Planning with Routing Pickup and Delivery Problem (ACLP+RPDP), which is mathematically modeled to tackle the challenges of pallet assembly, load balancing, and route planning in air transport.

The problem raised in this paper is innovative and applicable, and the proposed algorithm demonstrates effectiveness in solving the problem. However, there are several issues that need to be addressed:

\begin{itemize}

\item [1.] The experimental section lacks effectiveness analysis, where the specific contributions of each strategy are not thoroughly verified and analyzed through separate experiments.

\item[2.] The comparative experiment should include both well-known and novel algorithms for a comprehensive evaluation.

\item [3.] The algorithm is missing a complexity analysis, which is essential to better understand its efficiency and potential limitations.

\item [4.] The inclusion of more figures to illustrate the flow, principle, and results of the algorithm would greatly enhance the clarity and visual representation.

\item [5.] The format of the article needs further adjustment, particularly in addressing the existing large blank spaces on pages 7 and 10.

\end{itemize}

\subsection{Reviewer \#3}

This paper models the ACLP+RPDP of a real case, and develops a problem-solver. The experiments have demonstrated the contributions. However, there are some issues that should be addressed:

\begin{itemize}

\item [1:] In section 1, the authors claim that they solve an air cargo problem involving simultaneously APP, WBP, PDP, and TSP. But, there is not clear description of the importance. In other words, the motivation should be enhanced.

\item [2:] The literature review is simple. Although some studies focus on the same sub-problem in ACLPP, they still show differences in designing modelling and problem-solver. Apparently, authors should supplement them.

\item [3:] In section 5, the running time at each node is fixed. Whether different nodes have different requirements for running time.

\item [4:] In section 5.2.2, $\eta_1$\/ and $\eta_2$\/ are set by authors. It is better to study them by experiments.

\item [5:] The experimental settings are not complete.

\item [6:] To the best of our knowledge, the genetic algorithm is a good method for solving ACLPP. The method is not compared in the experiments.

\item [7:] With the increase of nodes, what impact will be brought in solving the problem. The experiments and analysis should be supplemented.

\end{itemize}




\section{\#3-b: Our Point-to-Point responses to the Reviewers}

In each subsection, we will respond to comments from each of the three reviewers.


\subsection{Reviewer \#1}

\paragraph{Revision:}
In section 2, Literature review in the first and second paragraphs is too simple. Some analysis needs to be given to introduce the research status.
\begin{center}
	\begin{minipage}{10cm}
		\textit{We appreciate this suggestion. We include in Section 2 a brief description of the six most recent and relevant articles on the research topic. We also commented on how the recent article by \cite{zhao2023} alerted us to potential performance problems in the solutions offered for air transport problems.}
		
	\end{minipage}
\end{center}


\paragraph{Revision:}
In the manuscript, any figures about algorithm simulation results cannot be found. Some figures and analysis are suggested to be presented to show the superiority of the used algorithm.
\begin{center}
	\begin{minipage}{10cm}
		\textit{As we commented in Sections 1 and 2 (see Table 2), we are dealing with a problem that has never been studied. Only \cite{LurkinSchyns2015} had addressed something similar but still quite different, as it did not produce a flight plan. In other words, no work has ever been published that has solved any instance of this problem. Therefore, developing a heuristic that presents solutions for it can be understood as a demonstration of superiority. On the other hand, we present the simulation results of our algorithm in Tables 12, 13, 14, and 15, and we now include Figures 6 and 7.}
	\end{minipage}
\end{center}


\paragraph{Revision:}
Also, any figures about experiments are not given in the manuscript. "By using a portable computer, our strategy quickly found practical solutions to a wide range of real problems in much less than operationally acceptable time." Could you please show the results with some figures?
\begin{center}
	\begin{minipage}{10cm}
		\textit{The solution for each ACLP+RPDP instance consists of 3 plans: the flight route; items that were delivered and shipped to each node on this route; and the allocation of items and packed contents on pallets at each node on this route. As it seems very difficult to express these plans through figures, we prefer to show two simpler indicators of the results obtained in each instance: objective function value and algorithm runtime. Some of these results were presented in Tables 12, 13, 14, and 15, and we now include Figures 6 and 7. Specifically, Figure 7 shows the flight plan of an instance in which $K=15$.}
	\end{minipage}
\end{center}


\paragraph{Revision:}
Too many old references are cited.
\begin{center}
	\begin{minipage}{10cm}
		\textit{We believe it is appropriate to maintain the old references, as they constitute a historical survey of the evolution of research in this field. On the other hand, if reviewers prefer to omit some of them, we don't see a problem with that. At the same time, we seem to cite all recent relevant work, similar to \cite{zhao2023}, which is an article less than a year old.}
	\end{minipage}
\end{center}


\subsection{Reviewer \#2}


\paragraph{Revision:}
The experimental section lacks effectiveness analysis, where the specific contributions of each strategy are not thoroughly verified and analyzed through separate experiments.
\begin{center}
	\begin{minipage}{10cm}
		\textit{
		It may not have been clear in the original text, but three strategies were essential for us to obtain a solution in the instances created:
		\begin{itemize}
		\item[(1)] consider $K<m$\/ so that each pallet received a preset destination;
		\item[(2)] on each tour, calculate a node-by-node solution;
		\item[(3)] at each intermediate node of the considered tour, reallocate the packed contents, minimising the torque on the aircraft.
		\end{itemize}
		Without adopting these three strategies, we would not be able to carry out any experiments, even in small instances. To make this clearer, we have included Figure 3, which describes the entire resolution process. The only strategy that allows different experiments and comparisons is the node-by-node solution. For this reason, we included Table 11, showing the effectiveness of each method for this problem.}
	\end{minipage}
\end{center}


\paragraph{Revision:}
The comparative experiment should include both well-known and novel algorithms for a comprehensive evaluation.
\begin{center}
	\begin{minipage}{10cm}
		\textit{
		As we are dealing with a new problem, we have no other solutions available in the literature. In fact, we only have well-known meta-heuristics for the node-by-node solution, in which the score/torque ratio is maximized. Table 11 shows the results obtained by NMO, ACO, GRASP, TS, and GA applied to this problem.}
	\end{minipage}
\end{center}


\paragraph{Revision:}
The algorithm is missing a complexity analysis, which is essential to better understand its efficiency and potential limitations.
\begin{center}
	\begin{minipage}{10cm}
		\textit{We appreciate this comment. In subsection 5.3, we include the time complexity analysis of the resolution process of an ACLP+RPDP instance.}
	\end{minipage}
\end{center}


\paragraph{Revision:}
The inclusion of more figures to illustrate the flow, principle, and results of the algorithm would greatly enhance the clarity and visual representation.
\begin{center}
	\begin{minipage}{10cm}
		\textit{We appreciate this suggestion. We reorganised the initial text of Section 5 and added Figure 3, which describes the entire resolution process. We also include Figures 6 and 7, which illustrate an important case. We consider that these changes facilitated the understanding of the proposed algorithm.}
	\end{minipage}
\end{center}


\paragraph{Revision:}
The format of the article needs further adjustment, particularly in addressing the existing large blank spaces on pages 7 and 10.
\begin{center}
	\begin{minipage}{10cm}
		\textit{These blank spaces appeared due to \LaTeX\/ formatting and should be eliminated in the final version.}
	\end{minipage}
\end{center}



\subsection{Reviewer \#3}


\paragraph{Revision:}
In section 1, the authors claim that they solve an air cargo problem involving simultaneously APP, WBP, PDP, and TSP. But, there is not clear description of the importance. In other words, the motivation should be enhanced.
\begin{center}
	\begin{minipage}{10cm}
		\textit{We added a new paragraph in Section 1, highlighting the motivations for ACLP+RPDP resolution: ``Inefficient air transport plans can lead to unnecessary costs, extra routes, longer distances, and incorrect destinations. Unbalanced cargo increases fuel consumption due to altered aircraft pitch angles, increasing the risk of weight- and balance-related accidents. Balancing cargo is crucial for safe aerial transportation, as an improperly positioned centre of gravity (CG) can result in dangerous take-off and landing conditions and stall recovery issues. Despite technological advancements, many airlines still rely on manual aircraft loading and balancing, which can lead to flight delays. A rational decision-making process is essential to avoid creating inefficient or unsafe transport plans, considering the high costs of fuel, maintenance, operation, outsourcing expenses, potential operational impairments, and safety risks due to unbalanced cargo. Solving this problem is vital for optimising strategic scores, saving time and effort in loading, ensuring safety and balance, correct pickups and deliveries, and finding the best routes considering fuel use due to potential cargo imbalance''.}
	\end{minipage}
\end{center}


\paragraph{Revision:}
The literature review is simple. Although some studies focus on the same sub-problem in ACLPP, they still show differences in designing modelling and problem-solver. Apparently, authors should supplement them.
\begin{center}
	\begin{minipage}{10cm}
		\textit{We appreciate this suggestion. We include more information about the main articles in this field of research, indicating their modelling, the scenarios considered, and the resolution methods adopted. It seems to us that this complementation has made the contribution of our work clearer.}
	\end{minipage}
\end{center}


\paragraph{Revision:}
In section 5, the running time at each node is fixed. Whether different nodes have different requirements for running time.
\begin{center}
	\begin{minipage}{10cm}
		\textit{We appreciate this suggestion, as we hadn't really thought about this possibility. We redid all our experiments, adopting, at each tour node, a runtime limit proportional to the load available for boarding. We found that there was a small improvement, which is why we incorporated this strategy into the resolution process.}
	\end{minipage}
\end{center}


\paragraph{Revision:}
In section 5.2.2, $\eta_1$\/ and $\eta_2$\/ are set by authors. It is better to study them by experiments.
\begin{center}
	\begin{minipage}{10cm}
		\textit{We appreciate this suggestion. In subsection 6.1 and Table 10, we explain how the {\bf irace} \citep{LopezIbanezManuel2016} tool was adopted to empirically define these parameters of Shims, described in Algorithm 6. There was indeed an improvement in the results, which were incorporated into Tables 12, 13, 14, and 15.}
	\end{minipage}
\end{center}


\paragraph{Revision:}
The experimental settings are not complete.
\begin{center}
	\begin{minipage}{10cm}
		\textit{Unfortunately, we do not understand this comment. Anyway, we have expanded Section 6, defining all the variables involved in the experiments carried out. If there is an unclear point, we ask the reviewer to please specify it.}
	\end{minipage}
\end{center}


\paragraph{Revision:}
To the best of our knowledge, the genetic algorithm is a good method for solving ACLPP. The method is not compared in the experiments.
\begin{center}
	\begin{minipage}{10cm}
		\textit{We appreciate this suggestion. We used DEAP (Distributed Evolutionary Algorithms in Python), an evolutionary computation framework \citep{DEAP_JMLR2012}, to compare GA as one of the heuristics in the node-by-node solution. In our first version, considering only weight and volume constraints, our implementation was able to solve problems of up to 100 items being allocated on 18 pallets in less than 10 seconds. When we included the item count constraint to prevent the same item from being allocated to more than one pallet, we were able to resolve only small problems (20 items on 2 pallets). We increased the number of items to 150 and the number of pallets to 18, but even adjusting the GA parameters for more generations (2,400) and more individuals (1,200), DEAP did not generate any viable solutions. The runtime for this attempt was about 10 minutes. These results are included in Table 11. On the other hand, we successfully used DEAP in developing a TSP heuristic for cases where $K>6$, as can be seen in subsubsection 6.2.2.}
	\end{minipage}
\end{center}


\paragraph{Revision:}
With the increase of nodes, what impact will be brought in solving the problem. The experiments and analysis should be supplemented.
\begin{center}
	\begin{minipage}{10cm}
		\textit{We appreciate this suggestion. We included tests for solving cases with $K>6$, maintaining the requirement $K < m$, necessary to previously define the destination of the pallets. The results are described in subsubsection 6.2.2 and in Figures 6 and 7.}
	\end{minipage}
\end{center}



\bibliographystyle{apalike} 

\bibliography{bib}


\end{document}
